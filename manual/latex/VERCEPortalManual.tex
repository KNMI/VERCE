\documentclass[english]{book}
\usepackage{pdfpages} 
\usepackage{tocloft}
\setlength{\cftchapnumwidth}{2em}
\cftsetindents{section}{2em}{2.4em}
\cftsetindents{subsection}{4.4em}{3.2em} 
\ifdefined\pdfpxdimen
   \let\sphinxpxdimen\pdfpxdimen\else\newdimen\sphinxpxdimen
\fi \sphinxpxdimen=.75bp\relax

\PassOptionsToPackage{warn}{textcomp}
\usepackage[utf8]{inputenc}
\ifdefined\DeclareUnicodeCharacter
 \ifdefined\DeclareUnicodeCharacterAsOptional
  \DeclareUnicodeCharacter{"00A0}{\nobreakspace}
  \DeclareUnicodeCharacter{"2500}{\sphinxunichar{2500}}
  \DeclareUnicodeCharacter{"2502}{\sphinxunichar{2502}}
  \DeclareUnicodeCharacter{"2514}{\sphinxunichar{2514}}
  \DeclareUnicodeCharacter{"251C}{\sphinxunichar{251C}}
  \DeclareUnicodeCharacter{"2572}{\textbackslash}
 \else
  \DeclareUnicodeCharacter{00A0}{\nobreakspace}
  \DeclareUnicodeCharacter{2500}{\sphinxunichar{2500}}
  \DeclareUnicodeCharacter{2502}{\sphinxunichar{2502}}
  \DeclareUnicodeCharacter{2514}{\sphinxunichar{2514}}
  \DeclareUnicodeCharacter{251C}{\sphinxunichar{251C}}
  \DeclareUnicodeCharacter{2572}{\textbackslash}
 \fi
\fi
\usepackage{cmap}
\usepackage[T1]{fontenc}
\usepackage{amsmath,amssymb,amstext}
\usepackage{babel}
\usepackage{times}
\usepackage[Bjarne]{fncychap}
\usepackage{sphinx}

\usepackage{geometry}

% Include hyperref last.
\usepackage{hyperref}
% Fix anchor placement for figures with captions.
\usepackage{hypcap}% it must be loaded after hyperref.
% Set up styles of URL: it should be placed after hyperref.
\urlstyle{same}


\addto\captionsenglish{\renewcommand{\figurename}{Fig.}}
\addto\captionsenglish{\renewcommand{\tablename}{Table}}
\addto\captionsenglish{\renewcommand{\literalblockname}{Listing}}

\addto\captionsenglish{\renewcommand{\literalblockcontinuedname}{continued from previous page}}
\addto\captionsenglish{\renewcommand{\literalblockcontinuesname}{continues on next page}}

\addto\extrasenglish{\def\pageautorefname{page}}

\setcounter{tocdepth}{2}

\titleformat{\chapter}[display]{\bfseries\Large}{\filright}{1ex}{}[]

\newcommand{\sphinxlogo}{\vbox{}}
\renewcommand{\releasename}{Release}
\makeindex

\begin{document}

\phantomsection\label{\detokenize{index::doc}}

\sphinxincludegraphics{{logo}.png}

\vspace{20mm}

{\huge{The VERCE portal – a user’s guide}}

\vspace{20mm}

\sphinxincludegraphics{{aquila_sim}.png}

\vspace{20mm}

\sphinxstylestrong{Version 2.1}

\vspace{5mm}

\sphinxstylestrong{April 2018}

\break

The VERCE project was funded by the EU as an EU infrastructure project,
involving a wide range of institutions from across the EU.

\vspace{20mm}

\scalebox{0.200000}{\sphinxincludegraphics[height=500\sphinxpxdimen]{{cineca}.png}} \hspace{5mm}
\sphinxincludegraphics{{emsc}.jpg} \hspace{5mm}
\sphinxincludegraphics[scale=0.4]{{edinburgh}.jpg}

\vspace{5mm}

\sphinxincludegraphics[width=2.0in]{{lmu}.png} \hspace{5mm}
\sphinxincludegraphics[width=1.2in]{{ingv}.jpg} \hspace{5mm}
\sphinxincludegraphics[width=2.5in,height=0.77500in]{{liverpool}.png}

\vspace{5mm}

\sphinxincludegraphics[width=2.5in,height=0.70139in]{{scai}.jpg} \hspace{15mm}
\sphinxincludegraphics[width=2.5in,height=0.80347in]{{knmi}.png}

\vspace{5mm}

\scalebox{0.800000}{\sphinxincludegraphics{{orfeus}.png}}

\vspace{20mm}

The following individuals have contributed material to this guide

T. Garth, F. Magnoni, R. Saleh, A. Spinuso, E. Casarotti, A. Gemund, S.
Hoon Leong, J. Holt, L. Krishner, A. Kraus, R. Filgueira, M. Aktinson,
H. Igel, A. Rietbrock


\sphinxtableofcontents

\chapter{Introduction to the VERCE platform}
\label{\detokenize{Section1::doc}}
\label{\detokenize{Section1:introduction-to-the-verce-platform}}
The VERCE portal is an online resource that allows large scale, 3D full
waveform simulations to be easily run on a variety of high performance
computers, and the associated large data sets to be managed easily. This
resource is designed to make the ever increasing potential of full
waveform seismological techniques available to a much wider spectrum of
seismology and earthquake science communities.

This guide provides a brief description of the science and technologies
behind the VERCE portal, and a step by step guide to using the portal.
The written description of each step is complemented by online tutorial
videos and presentations, and provides a stepping off point for users
who want to use other tools supported by VERCE such as ‘Obspy’, a python
toolbox for dealing with seismic data, and ‘Dispel4Py’, a python toolbox
designed to manage large data sets in seismology (and other
disciplines). The resources supported by VERCE should then enable a
seismologist of any specialism run large scale waveform simulations for
an area of interest, and deal with large seismological data sets.

VERCE \textendash{} Virtual Earthquake and seismology Research Community e-science
environment in Europe - is an EU infrastructure project extending from
2011 and 2015, and the development of the VERCE portal has been a
collaborative project involving a wide range of project partners from
across the EU. VERCE was a major contribution to the e-science
environment of the European Plate Observing System (EPOS), which is
presently supporting further developments and updates of the VERCE
platform.

\sphinxincludegraphics{{image15}.png}

\sphinxstylestrong{Figure 1.1:} A snapshot of a simulation of the Ml 5.2 Lunigiana
earthquake which occurred on the 21$^{\text{st}}$ June 2013 in Northern
Italy. Image produced by the INGV.



\chapter{Introduction to full waveform modeling}
\label{\detokenize{Section2:introduction-to-full-waveform-modeling}}\label{\detokenize{Section2::doc}}

\section{Why full waveform?}
\label{\detokenize{Section2:why-full-waveform}}
Much of the modern seismological methods used today rely on very little
of the information contained within the seismogram. Earthquakes can be
located using just the arrival time of the P- and S- waves, and this
same information from many earthquakes in a given area can be used to
invert for a travel time tomography. These methods however neglect the
vast majority of the data within the waveform.

The seismic waveforms recorded from an earthquake principally depend
upon the seismic structure of the medium they pass through (i.e. the
Earth), and the characteristics of the earthquake source itself. If this
waveform data is utilized, it could therefore lead to a much greater
resolution of seismic properties (e.g. velocity and attenuation) of the
Earth, as well as better models of earthquakes’ sources and ground
motion from potentially hazardous earthquakes.


\section{Calculating the wave field}
\label{\detokenize{Section2:calculating-the-wave-field}}
In order to use the full waveform to constrain a seismic wave speed
model of the Earth, we need a way of calculating the response of the 3D
velocity models we already have. The physics of how a seismic wave
propagates through the earth is well understood, and is described by the
wave equation.

A number of different methods exist to solve the wave equation (or its
constituent equations) in order to model the propagation of a waveform
from its source through a velocity model to a receiver. In order to do
this, the modelled area is broken up into a grid of points with given
seismic properties, and the motion of each point in that grid is
calculated at each time step.

The spacing of these points is controlled by the velocity model used and
the frequency of the seismic wave that is modelled, with tighter grid
spacing and a smaller time step being required to model higher frequency
seismic waves.

The time step and grid spacing are also dependent on the method used to
solve the wave propagation. The codes supported within the VERCE
platform use the spectral element methods (Komatitsch et al., 2005) to
solve the seismic wave equation, hence calculating the response of the
seismic wave-field to the velocity and attenuation parameters of the
input model. Details on how to ensure that the model is stable to the
frequencies required are given in section 8 of this guide.


\section{High performance computing in seismic waveform modeling}
\label{\detokenize{Section2:high-performance-computing-in-seismic-waveform-modeling}}
The main problem/issue with this sort of simulation is that calculating
the motion of a tight grid of points at thousands of time steps is
extremely computationally expensive. For this reason, both 2D and 3D
waveform models are run on high performance computers. These simulations
are executed in parallel, meaning that the simulation runs
simultaneously on many different processing cores. Simulations performed
through the VERCE platform can be run on a range of different super
computers from across Europe. The supercomputer pictured in figure 2.1
is hosted at SCAI Fraunhofer, Germany.

\begin{center}\raisebox{-0.5\height}{\sphinxincludegraphics{{image1}.jpg}}
\end{center}
\sphinxstylestrong{Figure 2.1:} One of the high performance computers which is
available through the VERCE platform is hosted at SCAI Fraunhofer,
Germany (\sphinxurl{https://www.scai.fraunhofer.de}).


\section{What do I need to run a simulation?}
\label{\detokenize{Section2:what-do-i-need-to-run-a-simulation}}
In order to run a full waveform seismic simulation you need to know the
following details:
\begin{enumerate}
\item {} 
\sphinxstylestrong{Velocity model}

In order to calculate the wave-field you need to know the velocity
structure of the area of interest. The velocity models already available
through the portal are all based on published travel time tomography
models, but it is also possible to load your own wave speed models into
the portal (see section 8).

\item {} 
\sphinxstylestrong{Mesh}

\end{enumerate}
\begin{quote}

The mesh is a grid of points within the volume that is to be modelled.
The mesh takes into account the changes in required resolution with
depth, but also the topography and bathymetry of the area being
modelled. Several meshes are already loaded into the portal for specific
regions. However, if you wish to run simulations in a new area, you will
need to create and submit a new mesh and velocity model. Details of how
to do this are given in section 8 of this guide.
\end{quote}
\begin{enumerate}
\setcounter{enumi}{2}
\item {} 
\sphinxstylestrong{Focal mechanisms}

\end{enumerate}
\begin{quote}

For any earthquake you want to model you must know the hypocentral
location as well as the full focal mechanism. The VERCE portal allows
you to download focal mechanisms from the gCMT catalogue and other local
earthquake mechanism catalogues (e.g. INGV catalogue of TDMT solutions),
but it is also possible to load in your own focal mechanisms as
described in section 8.
\end{quote}
\begin{enumerate}
\setcounter{enumi}{3}
\item {} 
\sphinxstylestrong{Station locations}

\end{enumerate}
\begin{quote}

Again, using the portal, users can select station locations by accessing
the information on international seismic networks through FDSN web
services. However, if there are any networks or station locations that
are not pre-loaded to the portal, you can also add your own station
locations (section 8).
\end{quote}

All of these are provided for certain study areas, such as the area we
consider in the simple example in part 5 of this guide.

However, If you are working on a study area that is not currently
supported in the VERCE portal, you will also need to create a new mesh,
velocity model, station locations and possible earthquake focal
mechanisms (although there are global catalogues supported in the
platform) and you can use them for bespoke simulations.



\chapter{Registering for the platform and certification}
\label{\detokenize{Section3:registering-for-the-platform-and-certification}}\label{\detokenize{Section3::doc}}
The VERCE platform can be accessed through a normal web browser. The
portal has been well tested with browsers such as \sphinxstyleemphasis{Mozilla Firefox} and
\sphinxstyleemphasis{Safari}, and should also work with \sphinxstyleemphasis{Google Chrome}. At this stage the
portal doesn’t support \sphinxstyleemphasis{Internet Explorer} or \sphinxstyleemphasis{Microsoft Edge.}


\section{Registering for the platform}
\label{\detokenize{Section3:registering-for-the-platform}}
Registering for the portal is exceptionally easy. Simply go to the VERCE
portal website on the link below and click the \sphinxincludegraphics{{image11}.jpg}icon. Fill in
your details ensuring that you use your university email address (i.e.
your academic email address) if at all possible. This can take one or
two working days to be processed so please leave yourself time for this.

\sphinxurl{https://portal.verce.eu/home}

\sphinxincludegraphics{{image22}.png}

\sphinxstylestrong{Figure 3.1:} The VERCE sign up page


\section{Getting a certificate}
\label{\detokenize{Section3:getting-a-certificate}}
In addition to registering for the platform you must also get an
e-science certificate from the relevant authority. This is an
internationally recognised certification scheme that will allow you to
submit simulations to a wide range of supercomputers. Without an
e-science certificate you will be able to log into the portal website,
but you will not be able to submit jobs to any of the super computers,
or access any data through the portal.

The procedure for getting an e-science certificate depends on the where
you are based as the e-science certificates are distributed by national
certification authorities. First you must request an e-science
certificate from your national provider, following the instructions on
the relevant website:
\begin{itemize}
\item {} 
For the UK: \sphinxurl{http://www.ngs.ac.uk/ukca/certificates}

\item {} 
For Germany:
\sphinxurl{https://pki.pca.dfn.de/grid-root-ca/cgi-bin/pub/pki?RA\_ID=101}

\item {} 
For the Netherlands: \sphinxurl{http://ca.dutchgrid.nl/}

\item {} 
For France: \sphinxurl{https://igc.services.cnrs.fr/GRID2-FR}

\item {} 
For Italy: \sphinxurl{http://security.fi.infn.it/CA}

\end{itemize}

You then have to go to an administrator (usually in your university) who
confirms who you are, and (hopefully) approves the certificate. This can
also take a day or two, so please leave time for this.

It is important to back up this certificate in a different location to
the host computer (i.e. the computer that you used to request and
download the certificate). Your local certificating authority will
provide full details of how to do this.


\section{Installing your certificate in your browser}
\label{\detokenize{Section3:installing-your-certificate-in-your-browser}}
To make the next stages of registration easier it is recommended that
the users install their new certificate to the browser. Ideally this
should be done on the computer you are most likely to be using the VERCE
portal from. The certificate must first be exported from the certificate
manager (see instructions from your national certificating authority).
Both the ‘Private Key’ and the Certificate should be exported in
‘PKCS\#12’ format. The certificate is exported to the local machine, and
is protected with a new password.

To install the certificate on Mozilla you must first select
\sphinxstyleemphasis{‘Options/Preferences’} from the menu button (\sphinxincludegraphics{{image32}.png}) in the top right
of the browser. Select the ‘\sphinxstyleemphasis{Advanced}’ tab from the left-hand panel
and then the ‘Certificates’ tab under the ‘Advanced’ menu. Finally
select ‘\sphinxstyleemphasis{View Certificates’} and click \sphinxstyleemphasis{‘Import’} to upload your
certificate from its location on your local machine using the new
password set when exporting the certificate.

Other browsers can be used (see start of section for browser options),
but the procedure for uploading the certificate will vary. You must then
ensure that you use this browser for the validation steps outlined
below, and ideally when you log in to the portal.


\section{Registering for super computing and data resources}
\label{\detokenize{Section3:registering-for-super-computing-and-data-resources}}
The VERCE portal and iRODS are currently hosted by the ‘SCAI Fraunhofer’
supercomputer, in Germany.

To register for SCAI and iRODS, please send the information listed below to André Gemünd (\sphinxhref{mailto:andre.gemuend@scai.fraunhofer.de}{andre.gemuend@scai.fraunhofer.de}) at
SCAI Fraunhofer, and request to be registered for the VERCE portal.
André will be able to give you an account on ‘SCAI’ and ‘iRODS’ that
will allow you to calculate and manage waveforms respectively. ‘iRODS’
is a suite of data management software that is embedded within the VERCE
platform, and allows you to easily access your data regardless of where
you submitted your simulation.
\begin{itemize}
\item {} 
First Name

\item {} 
Last Name

\item {} 
Nationality

\item {} 
Affiliation

\item {} 
Professional Address (including country)

\item {} 
Telephone Number

\item {} 
Email address

\item {} 
Certificate DN (distinguished name, also called subject) of your
certificate

\end{itemize}


\section{Creating and uploading proxy certificates}
\label{\detokenize{Section3:creating-and-uploading-proxy-certificates}}
Once you have got your national e-science certificate, you will need to
load a proxy certificate up to the portal to allow you to access data
and submit jobs to the available supercomputers. A proxy certificate is
essentially a copy of the full certificate that will expire after a
short time period (usually 24 hours). This allows you to upload and use
your certificate, while the limited life span of the proxy minimises the
risk of the certificate falling into the wrong hands.

Currently you can create a Proxy certificate using either the
GSISSH\_Term tool or the command line MYPROXY tools, both described
below. We hope to provide a proxy certificate tool in the near future.


\subsection{MYPROXY Tools}
\label{\detokenize{Section3:myproxy-tools}}
To run simulations on the VERCE Portal it is necessary to have stored
your credentials (a proxy certificate) in a MyProxy repository so that
it’s available for download through the portal when needed. To create a
proxy certificate, you need to have a user certificate (from your
national provider, listed above) and a private key file (with a .PEM
format).

If your certificate is in p12 format, the relevant certificate format
and private key file can be created as below. Run the following commands
in the command line:

\fvset{hllines={, ,}}%
\begin{sphinxVerbatim}[commandchars=\\\{\}]
openssl pkcs12 \PYGZhy{}clcerts \PYGZhy{}nokeys \PYGZhy{}out usercert.pem \PYGZhy{}in cert.p12

openssl pkcs12 \PYGZhy{}nocerts \PYGZhy{}out userkey.pem \PYGZhy{}in cert.p12
\end{sphinxVerbatim}

The above commands should generate the files usercert.pem and
userkey.pem. Once this is done then to protect your keys you would need
to run the following commands:

\fvset{hllines={, ,}}%
\begin{sphinxVerbatim}[commandchars=\\\{\}]
chmod 644 usercert.pem

chmod 400 userkey.pem
\end{sphinxVerbatim}

Before you could use the below proxy tools you will need first to
perform the following:
\begin{itemize}
\item {} 
Create a “. globus” folder in your home directory and then copy the
files usercert.pem and userkey.pem to this particular folder

\item {} 
Install Java Runtime Environment (JRE) 1.7 or higher
\begin{quote}

For the GSISSH\_term tool you should also do the following steps:
\end{quote}

\item {} 
Download the “Java Cryptography Extension (JCE) Unlimited Strength
Jurisdiction Policy Files”

\item {} 
Extract the two jar files, “local\_policy.jar” and
“US\_export\_policy.jar”, and copy them to \{JRE\_HOME\}/lib/security

\end{itemize}


\subsubsection{GSISSH\_Term Tool}
\label{\detokenize{Section3:gsissh-term-tool}}
The GSISSH\_Term is a Java based application supported on most
platforms. It is currently available for download on
\sphinxurl{https://www.lrz.de/services/compute/grid\_en/software\_en/gsisshterm\_en/}
and can be installed either as a desktop application or as a Java
webstart application.

To run GSISSH-Term as a Java webstart application, you need to have Java
webstart (javaws) installed on your machine. This should already be
included in the Java Runtime Environment (JRE) for Java SE 7.

Once the GSISSH-Term application has ran successfully then you can
launch “MyProxy Tool” by selecting “MyProxy Tool” option from the
“Tools” menu. A new popup window should appear as shown Figure 3.5.1.1.

\sphinxincludegraphics{{image4}.jpg}

\sphinxstylestrong{Figure 3.5.1.1:} Launching MyProxy Tool on GSISSH\_Term
application.

With MyProxy Tool you can upload, check and remove your credential
to/from a MyProxy server. The tool also supports the generation and
upload of voms-enabled proxy.

To create and store a proxy certificate in a MyProxy server, do the
following steps:
\begin{itemize}
\item {} 
Launch MyProxy Tool on a machine where your Grid credentials are
located.

\item {} 
Select “Create and upload proxy” from the dropdown list.

\item {} 
In the “MyProxy Server Information” panel enter:
\begin{itemize}
\item {} 
a URL of MyProxy server,

\item {} 
a port number to connect to MyProxy server,

\item {} 
a lifetime span for your proxy certificate and

\item {} 
a username you could use later to retrieve or download your
credentials from MyProxy server.

\end{itemize}

\item {} 
In the “Local Certificate” panel:
\begin{itemize}
\item {} 
choose your certificate format from the dropdown list,

\item {} 
for PEM format as mentioned above place both usercert.pem and
userkey.pem in a folder named “.globus” which should be located
within your home directory.

\item {} 
For passphrase enter your grid-proxy passphrase.

\end{itemize}

\item {} 
Click on the “Create and Store” button.

\item {} 
Once the connection to MyProxy server has been established then you
will be prompted to enter a new passphrase which you will need to use
later along with other details to access your credentials on MyProxy
server. It is recommended to use a passphrase that is different to
your grid-proxy passphrase

\end{itemize}


\subsubsection{MYPROXY command line tool}
\label{\detokenize{Section3:myproxy-command-line-tool}}
The MYPROXY tools can then be installed as by running the following
commands in the command line of a Linux or mac machine:

\fvset{hllines={, ,}}%
\begin{sphinxVerbatim}[commandchars=\\\{\}]
sudo apt\PYGZhy{}get install myproxy

sudo apt\PYGZhy{}get install voms\PYGZhy{}clients
\end{sphinxVerbatim}

Installing the packages below will allow you to manage your certificate
(from your national certificating authority). More details on this are
given at \sphinxurl{https://wiki.egi.eu/wiki/EGI\_IGTF\_Release}. To install the
relevant packages:
\begin{enumerate}
\item {} 
Add the following line to your dpkg sources (sources.list file):

\fvset{hllines={, ,}}%
\begin{sphinxVerbatim}[commandchars=\\\{\}]
\PYGZsh{}\PYGZsh{}\PYGZsh{}\PYGZsh{} EGI Trust Anchor Distribution \PYGZsh{}\PYGZsh{}\PYGZsh{}\PYGZsh{}

deb http://repository.egi.eu/sw/production/cas/1/current egi\PYGZhy{}igtf core
\end{sphinxVerbatim}

\item {} 
Run the following to add the EUGridPMA PGP key:

\fvset{hllines={, ,}}%
\begin{sphinxVerbatim}[commandchars=\\\{\}]
sudo wget \PYGZhy{}q \PYGZhy{}O \textendash{}\PYG{l+s}{{}`https://dist.eugridpma.info/distribution/igtf/current/GPG\PYGZhy{}KEY\PYGZhy{}EUGridPMA\PYGZhy{}RPM\PYGZhy{}3\PYGZbs{}\textbar{} }\PYG{l+s+si}{\PYGZlt{}https://dist.eugridpma.info/distribution/igtf/current/GPG\PYGZhy{}KEY\PYGZhy{}EUGridPMA\PYGZhy{}RPM\PYGZhy{}3\textbar{}\PYGZgt{}}\PYG{l+s}{{}`\PYGZus{}\PYGZus{}}
    sudo apt\PYGZhy{}key add \textendash{}
\end{sphinxVerbatim}

\item {} 
Populate the cache and install the meta-package

\fvset{hllines={, ,}}%
\begin{sphinxVerbatim}[commandchars=\\\{\}]
sudo apt\PYGZhy{}get update

    sudo apt\PYGZhy{}get install ca\PYGZhy{}policy\PYGZhy{}egi\PYGZhy{}core
\end{sphinxVerbatim}

\end{enumerate}

To store a credential in the MyProxy repository, run the \sphinxstyleemphasis{myproxy-init}
command on a computer where your Grid credentials are located. For
example:

\fvset{hllines={, ,}}%
\begin{sphinxVerbatim}[commandchars=\\\{\}]
myproxy\PYGZhy{}init \PYGZhy{}s myproxy.lrz.de
\end{sphinxVerbatim}

This will prompt you for your grid-proxy passphrase and then for a new
passphrase for accessing your credentials from the MyProxy server. It is
recommended to use a passphrase that is different to your grid-proxy
passphrase.

By default, MyProxy uses your local Unix username to store your
credentials and the proxy certificate is stored for 7 days in the
MyProxy server. However, you can change this and register a proxy
certificate for a specific number of hours using the -c option.

For example, running the following command will register a proxy
certificate for 2 hours

\fvset{hllines={, ,}}%
\begin{sphinxVerbatim}[commandchars=\\\{\}]
myproxy\PYGZhy{}init \PYGZhy{}c 2 \PYGZhy{}s myproxy.lrz.de
\end{sphinxVerbatim}

To download your certificate, go to the ‘Security’ tab on the VERCE
Portal and click on the ‘Download’ button under the ‘Certificate’ panel
then under ‘\sphinxstylestrong{Hostname’} enter the MyProxy server (e.g.
myproxy.lrz.de) and your MyProxy username and passphrase. This is
described in more detail in section 3.5.2 (below).

For more details of MyProxy commands, see

\sphinxurl{http://toolkit.globus.org/toolkit/docs/4.0/security/myproxy/user-index.html}


\subsection{Uploading a proxy certificate to the VERCE platform}
\label{\detokenize{Section3:uploading-a-proxy-certificate-to-the-verce-platform}}
Once you have created you proxy certificate using one of the methods
above, you need to load the certificate into the VERCE platform in order
to be able to submit jobs and access the data.

To do this you need to go to the ‘Security’ page. On this page click the
\sphinxincludegraphics{{image5}.jpg} button which will display the proxy certificate upload panel as
shown below in Figure 3.5.

\sphinxincludegraphics{{image6}.jpg}

\sphinxstylestrong{Figure 3.5:} The proxy certificate upload panel.

Here you need to enter the address of the institution hosting your proxy
certificate in the \sphinxstylestrong{‘Hostname’} box (e.g. myproxy.lrz.de). The
username and password you set for your proxy certificate must then be
entered in the \sphinxstylestrong{‘Login’} and \sphinxstylestrong{‘Password’} boxes respectively.
Clicking the \sphinxincludegraphics{{image5}.jpg}button then adds the proxy certificate to the
portal, allowing you to access the high performance computing (HPC)
resources and data.

The proxy certificate will only be valid for up to 24 hours, so you will
need to repeat this process of creating and uploading the proxy
certificate every time you wish to use the VERCE portal to run
simulations or access data.


\subsection{Certificate Association}
\label{\detokenize{Section3:certificate-association}}
Once you have successfully uploaded your certificate, you must associate
the certificate with the platform (verce.eu) and any resources you
intend to use in this session (e.g. supercomputing resources). The proxy
certificate will then authenticate you as a user, and allow you to
access the HPC and memory resources.

First navigate to the ‘Security’ tab. Here you will see details of the
proxy certificate you have just uploaded as shown below. Click the
‘Associate to VO’ button to bring up the page shown in figure 3.6. You
can now select the resource you wish to associate your proxy certificate
to from the drop-down menu located below your certificate details. You
can see in both figure 3.5 and figure 3.6 that the proxy certificate
here is associated to verce.eu, SuperMUC and SCAI\_Cluster2.

\sphinxincludegraphics{{image7}.jpg}

\sphinxstylestrong{Figure 3.5:} The proxy certificate loaded into the certificates page
of the portal. This proxy certificate has been associated with SCAI computing resources.

\sphinxincludegraphics{{image8}.jpg}

\sphinxstylestrong{Figure 3.6}: The certificate association page. The resources to
which the certificate may be associated are listed in the dropdown box
shown in the bottom right of the figure.


\chapter{A tour of the VERCE platform}
\label{\detokenize{Section4:a-tour-of-the-verce-platform}}\label{\detokenize{Section4::doc}}
In this section we will introduce the different parts of the VERCE
platform, and show the models and meshes that are already loaded into
the VERCE portal.


\section{Welcome \& News tabs}
\label{\detokenize{Section4:welcome-news-tabs}}
\sphinxincludegraphics{{image12}.jpg}

The welcome tab of the portal gives a very brief overview of the portals
uses, while the news tab gives details of recent significant
earthquakes. Other news such as upcoming training events, and
publications related to the VERCE platform may also be shown here.


\section{Security tab}
\label{\detokenize{Section4:security-tab}}
\sphinxincludegraphics{{image22}.jpg}

We have already used many of the features available in this tab in order
to upload our proxy certificate. The main security page though gives an
overview of how to register for and get certificated for the platform,
as covered in section 3 of this guide. The main tools you need to be
aware of are the \sphinxstylestrong{‘MyProxy tool’} (section 3.4.1) and the
\sphinxstylestrong{‘Certificate’} upload tab (section 3.4.3).


\section{Forward Modelling tab}
\label{\detokenize{Section4:forward-modelling-tab}}
\sphinxincludegraphics{{image3}.jpg}

The ‘Forward Modelling’ tab is the main feature of the portal, and it is
from here that you can setup and run full waveform simulations, and
analyse the obtained output products. This section is divided in five
sub-tabs that allow the user to access the different steps of the
simulation and analysis procedure.

Jobs can be run from the \sphinxstyleemphasis{‘Simulation’} tab shown in figure 4.1 below.
On the right hand side of this panel, the code used for the simulations
can be selected from the drop down menu ‘Solvers’: so far both a code
for local/regional 3D simulations - SPECFEM3D\_Cartesian (Peter et al.,
2011; see Section 5) \textendash{} or a code for regional/global 3D simulations \textendash{}
SPECFEM3D\_GLOBE (Tromp, Komatitsch, and Liu 2008; see section 6) \textendash{} can
be selected. Then, existing pre-loaded mesh and associated velocity
model for different areas in the world can be selected from the drop
down menus ‘Meshes’ and ‘Velocity Model’ respectively. Earthquake
sources and seismic stations can be selected from the catalogues that
are pre-installed into the portal under the ‘Earthquakes’ and ‘Stations’
tabs respectively. This process is described in more detail later in
section 5 and 6 of this guide.

Alternatively, you can add your own mesh and velocity model using the
blue link below the drop down boxes in figure 4.1. You can then add your
own earthquake focal mechanisms and station locations. Details on how to
create and submit a more advanced bespoke job like this are given in
section 8 of this guide.

The left hand side of the panel shows a summary map of the area you are
running your model for, currently showing the default view of Europe.
The map also shows details of existing geological maps, hazard maps and
fault traces. The relative weight of these can be adjusted using the
drop down menu from the layer info button in the top right of the map
area.

\sphinxincludegraphics{{image41}.jpg}

\sphinxstylestrong{Figure 4.1:} The forward modeling interface ‘Simulation’ page.

From the \sphinxstyleemphasis{‘Download’} tab users can download observed seismograms from
the EIDA data archive corresponding to a specific earthquake selected
for simulations. These data cab be used in the subsequent procedure to
calculate misfit with respect to synthetic seismograms. Details in
section 7.

Moreover, both observed and synthetic seismograms can be processed
before comparison using the features under the \sphinxstyleemphasis{‘Processing’} tab and
quantitative misfit calculation can be performed in the section
\sphinxstyleemphasis{‘Misfit’}. All the results of simulations and analyses can be accessed
from the ‘Results’ tabs. Details are reported in section 7 of this
guide.


\section{Provenance tab}
\label{\detokenize{Section4:provenance-tab}}
\sphinxincludegraphics{{image51}.jpg}

This tab gives access to the provenance explorer GUI, which allows the
methods assumptions and inputs that have lead to a given synthetic
output to be easily summarized. An example of the provenance GUI is
shown below (figure 4.2)

\sphinxincludegraphics{{image62}.png}

\sphinxstylestrong{Figure 4.2:} An example displayed from the Provenance Explorer GUI
(taken for Atkinson et al 2016).


\section{File Manager tab}
\label{\detokenize{Section4:file-manager-tab}}
\sphinxincludegraphics{{image71}.jpg}

The file manager tab gives a access to the files that are available to
the user. The files are sorted by model run. Examples of using the
functionality of this tab are given in section 7.


\section{IRODS tab}
\label{\detokenize{Section4:irods-tab}}
\sphinxincludegraphics{{image81}.jpg}

The iRODS tab gives direct access to the iRODS data structure, and
allows the data to be managed and potentially downloaded. Examples of
using the functionality of this tab are given in section 7.


\section{The Meshes and models already uploaded}
\label{\detokenize{Section4:the-meshes-and-models-already-uploaded}}
Currently there are several meshes and velocity models pre-loaded for
Italy, and a mesh pre-loaded for the Maule area of Central Chile. They
can be used for running 3D simulations at local/regional scale with the
code SPECFEM3D\_Cartesion, as explained in section 5 of this guide.

The frequency to which the seismic wave-field can be simulated is
controlled by the time step of the model, the spacing of grid points
within the mesh and ultimately by the values of wave velocities in the
corresponding model. For this reason, there is a maximum frequency (or
minimum period) of waveform that a given pair of mesh and velocity model
can support. This minimum period (maximum frequency) resolvable is shown
below (figure 4.3) for each of the combinations mesh-wavespeed model
currently available in the VERCE portal.

Other details of these meshes such as the suggested time step (suggest
DT) to make each model stable, the number of points in the mesh (Num. of
HEX), and the approximate time that a 1 minute simulation would take if
it was run on 100 cores (CPU time) are shown in figure 4.4. The UTM zone
for each of the meshes is also shown as this should be specified to run
the simulation and can be useful when using the output data.

Finally figure 4.5 gives details of the velocity models that are
uploaded, along with the meshes, to the VERCE portal. The minimum and
maximum P-wave and S-wave velocities are given as these are required to
calculate the grid spacing and time step needed to resolve a given
frequency of seismic wave.

\sphinxincludegraphics[scale=0.5]{{image9}.jpg}

\sphinxstylestrong{Figure 4.3:} The mesh and velocity model combinations currently
available through the VERCE portal, and the period to which the
wave-field can be resolved in each of these mesh-model combinations.

\sphinxincludegraphics[scale=0.5]{{image10}.jpg}

\sphinxstylestrong{Figure 4.4:} Details of the meshes currently available through the
VERCE platform.

\sphinxincludegraphics[scale=0.5]{{image111}.jpg}

\sphinxstylestrong{Figure 4.5:} Details of the velocity models that are currently
available through the VERCE platform.


\chapter{A SPECFEM3D\_Cartesian simulation example}
\label{\detokenize{Section5::doc}}\label{\detokenize{Section5:a-specfem3d-cartesian-simulation-example}}
SPECFEM3D\_Cartesian is one of the principle solvers used in the VERCE
portal. It is designed to run waveform simulations on local to regional
scales, where local variations in bathymetry or topography may be
significant but large-scale features, such as the Earth’s curvature, may
be reasonably ignored. The Cartesian version of SPECFEM requires a
pre-made mesh that includes features of the area or region such as the
topography. These meshes can be produced by third party programs such as
GEOCUBIT (Casarotti et al., 2008), and are often refined for a
particular velocity model from a local or regional scale tomography for
instance. These meshes can be relatively complex to produce. However, in
the VERCE platform, a range of pre-loaded meshes is available. Users can
also upload their own mesh and velocity model for a region of interest.
This more advance use is described in chapter 8.

In the following section we will describe a step-by-step example of the
set up of a forward simulation with SPECFEM3D\_Cartesian.


\section{Preparing the portal}
\label{\detokenize{Section5:preparing-the-portal}}
Before you are able to run a simulation you must log in to the VERCE
portal, and upload a proxy certificate as described in section 3 of this
guide. Once you have done this, you should be able to use all parts of
the portal for the next 24 hours (or the lifetime of your proxy
certificate).


\section{Selecting a solver, mesh and velocity model}
\label{\detokenize{Section5:selecting-a-solver-mesh-and-velocity-model}}
The waveform simulations are run from the ‘Forward Modelling’ tab of the
VERCE portal, shown in figure 5.1. First you must select the ‘Solver’
tab from the top of the forward modelling panel. In the first drop down
menu you must select the solver. This is the code that will perform the
full waveform simulation. Currently the VERCE platform supports
SPECFEM3D\_Cartesian, which is designed to simulate waveforms on the
local/regional scale, and SPECFEM3D\_GLOBE, designed for 3D simulations
in the whole Earth. Specify one of the two codes by selecting
SPECFEM3D\_Cartesian or SPECFEM3D\_GLOBE in the drop down menu labelled
‘Solvers’. For this example we select SPECFEM3D\_Cartesian. Once you
have done this you will see that the right hand side of the panel is
populated with the input parameters for the selected code, which are
categorised into ten groups.

You can now select the area that you wish to run a simulation for from
the drop down menu labelled ‘Meshes’. Once you have selected the
relevant mesh, the map on the left of the ‘Forward Modelling’ panel will
zoom to the area concerned, and the area the mesh covers will be
outlined with a black box as shown below, in Figure 5.1.

\sphinxincludegraphics{{image13}.jpg}

\sphinxstylestrong{Figure 5.1:} Selecting a solver and mesh for Northern Italy. On the
left panel the colours show the local geology, and known faults are
plotted in black. On the right panel the drop down menu showing the
meshes that are currently loaded in is shown. The input parameters for
the solver can also be seen at the bottom of the right panel.

Once you have selected a solver and a mesh you can then select a
velocity model for the given area. Most of the meshes currently only
have one velocity model associated with them. But in theory it is
possible to have more than one velocity model for each mesh, as long as
the area of the velocity model covers at least the mesh dimensions. This
could for instance allow different tomographic models to be compared.


\section{Specifying the input parameters for SPECFEM3D\_Cartesian}
\label{\detokenize{Section5:specifying-the-input-parameters-for-specfem3d-cartesian}}
The input parameters for SPECFEM3D\_Cartesian are broken up into 10
categories, which are briefly described in Appendix 1. For a basic
simulation, many of the parameters can be left at the default setting
within the portal, but it is important to understand the meaning of
these input flags for more advanced uses. For explanation of the use of
the flags you can simply hold your curser above the question mark to the
right of the flag or variable for a brief description. For a fuller
description of these parameters please refer to Appendix 1, or for full
details refer to the manual of SPECFEM3D\_Cartesian.

It is very important that you check the parameters in Group 0,
especially insuring that the number of processors (NPROC), the time step
(DT) used in the simulation, and length of the simulation are all set
correctly. In particular, for each mesh/velocity model pair there is a
maximum value of DT above which the simulation becomes unstable (see
section 8.2.1 for details); thus for all your simulations we suggest to
use the best values of DT for each mesh/model pair reported in Figure
4.4.

Other parameters in Group 1-9 can be left as default for this exercise,
but also allow you to specify details of the simulation. While the
platform gives you as much flexibility as possible to vary these
parameters, the user must ensure that reasonable values are used,
otherwise the simulation may not run, or may not produce reasonable
results. The aspects of the simulation controlled by each group are
described briefly below.


\begin{savenotes}\sphinxattablestart
\centering
\begin{tabular}[t]{|\X{15}{100}|\X{85}{100}|}
\hline
\sphinxstyletheadfamily 
\sphinxstylestrong{Group}
&\sphinxstyletheadfamily 
\sphinxstylestrong{Description}
\\
\hline
\sphinxstylestrong{Group 0}
&
Contains the main flags of the code that controls the number of cores used on the HPC resources, the simulation duration and the time step value. GPU mode will be supported in the near future.
\\
\hline
\sphinxstylestrong{Group 1}
&
Controls the type of simulation that is run, and is currently limited to a forward simulation of an earthquake source. In the near future the platform will support the adjoint simulations allowed by the code.
\\
\hline
\sphinxstylestrong{Group 2}
&
Contains the details of the projection, which are set by default when you select the appropriate mesh.
\\
\hline
\sphinxstylestrong{Group 3}
&
Contains details of how attenuation is accounted for in the simulation. If the attenuation flags are left unchecked (as is the default), an elastic simulation will be run. If the flags are checked, attenuation is estimated from Olsen attenuation (Olsen et al., 2003). For this example we run an elastic case, and so leave the attenuation flags unchecked.
\\
\hline
\sphinxstylestrong{Group 4}
&
Allows the type of boundary conditions at the edge of the model area to be set.
\\
\hline
\sphinxstylestrong{Group 5}
&
Contains details of how the waveforms of the simulation are output.
\\
\hline
\sphinxstylestrong{Group 6}
&
Contains details of the seismic source that is implemented.
\\
\hline
\sphinxstylestrong{Group 7}
&
Contains details of how the movies of the simulation are output.
\\
\hline
\sphinxstylestrong{Group 8 \& 9}
&
Contain more advance parameters that are not used in the examples presented here. Advance users who wish to use these parameters should refer to the SPECFEM3D\_Cartesian manual.
\\
\hline
\end{tabular}
\par
\sphinxattableend\end{savenotes}

\begin{center}Table 5.1: Summary of the groups for SPECFEM3D\_Cartesian
\end{center}
The final set up of the workflow is shown in Figure 5.2.

\sphinxincludegraphics{{image23}.jpg}

\sphinxstylestrong{Figure 5.2:} Final setup of the workflow for the simple example. The
area of Northern Italy which is to be simulated is outlined by the black
box.


\section{Selecting earthquakes}
\label{\detokenize{Section5:selecting-earthquakes}}
Next the earthquakes to be simulated are defined in the ‘Earthquakes’
tab at the top right of the ‘Forward Modelling’ page. There are a number
of earthquake catalogues pre-loaded into the VERCE platform, including
the global CMT catalogue, which provides a starting data set for any
area that we have a model and mesh for. It is also possible to load in
your own bespoke catalogue of earthquake moment tensors, as described in
greater detail in section 8.4. Given that the model and mesh selected in
our example is in Northern Italy, we can however use the INGV focal
mechanism data set, which is likely to have a larger range of events
down to smaller magnitudes.

The earthquake catalogue you wish to search is selected using the drop
down menu at the top of the ‘Earthquakes’ tab. You can then search for
earthquakes in a range of magnitudes, depths and time. In this example
we have searched for all earthquakes from magnitude Mw 4.0 \textendash{} 9.0, at up
to 100000 m depth that occurred within the model and mesh area in the
year 2012. The earthquake of interest can then be selected either by
ticking the box next to the earthquake in the list in the bottom right
of the panel, or by selecting the location of the earthquake on the map
in the left of the panel. It is also possible to select multiple events
from this page. This will then submit the same number of jobs as events
that you have selected, and produce waveforms for each of them. This
would then allow multiple events to be used in an inversion for
instance.

\sphinxincludegraphics{{image31}.jpg}

\sphinxstylestrong{Figure 5.3:} The earthquake selection page of the forward modelling
tool. Events are shown from the INGV catalogue. The locations of the
events are shown in the summary map on the left, and details of the
events are shown in the bottom right hand part of the panel. The event
or events to be modelled can be selected from either of these panels.

For now though, just submit one event. In Figure 5.3 an earthquake in
the centre of the mesh has been selected so that we see a nice clear
waveform on all stations. You can select any event you are interested
in, but be aware that events close to the limits of the mesh may be more
greatly affected by the absorbing boundary conditions at the edges of
the model.


\section{Selecting stations}
\label{\detokenize{Section5:selecting-stations}}
The seismometers where you want to simulate the synthetic seismogram can
then be selected under the ‘Stations’ tab on the right hand side of the
‘\sphinxstyleemphasis{Forward Modelling’} panel. The portal is configured to output the
synthetic waveforms at points where real seismometers exist so that the
synthetic waveforms can be directly compared to the observed waveforms
recorded at these stations.

There are many seismic networks loaded into the VERCE portal that can be
used. To see all of the stations that are available within the mesh and
model area you can simply select one ‘Provider’ and ‘*’ (i.e., any
network) in the drop down box at the top of the ‘Stations’ panel.
Alternatively you can select a given network you are interested in (or
have the data for), for instance the INGV network (network code IV). You
can then manually select the stations you are interested in by clicking
on the stations in the map view, or selecting the tick box next to the
station information in the right hand panel. All of the stations in this
list can be selected by selecting the tick box at the top of the list.
See Figure 5.4.

\sphinxincludegraphics{{image42}.jpg}

\sphinxstylestrong{Figure 5.4:} The station selection panel. Stations are shown inside
the area of the model and mesh (shown by the black box) for the INGV
network. The stations (shown in blue) can be selected individually from
the map or from the station list. All the available stations can be
selected by ticking the tick box in the title bar (labelled 0/0 when no
stations are selected).

While selecting a large number of stations will not affect the overall
time taken or computational cost of the simulation, the more stations
you select the longer it will take to move the simulation output to
memory where you can then access it. For large simulations it is most
efficient therefore to output seismograms for all the stations that you
may be interested in.


\section{Submitting the job}
\label{\detokenize{Section5:submitting-the-job}}
You can then select the workflow you wish to use and submit your job.
Currently one workflow is available, for the super computer at SCAI
Fraunhofer
(\sphinxhref{http://www.hpc.cineca.it/)}{https://www.scai.fraunhofer.de)}. Other
workflows will be introduced, allowing users to run the simulation on a
variety of HPC resources across Europe. Each workflow is configured to a
different machine, and some machines have more than one workflow
available on them. You can then select the relevant workflow (here we
have selected ‘SCAI\_mpi\_SPECFEM\_PRODUCTION’) for the HPC resource you
wish to submit to, and enter a name and description of the run. Please
note that the name of the model must be 20 characters or less, and can
only consists of letters, numbers and decimals. Other characters are not
accepted.

If you have selected more than one event to simulate, you may select the
‘Process the events in parallel’ box. This submits each of the events to
the queue of the resources you have selected as separate jobs, rather
than running the jobs in serial (one after another). This can speed up
the process of simulating several earthquake events for the same
velocity model.

\sphinxincludegraphics[scale=0.5]{{image52}.png}

\sphinxstylestrong{Figure 5.5:} Available workflow from the top drop down menu of the
‘submit’ tab of the forward modelling page.


\section{Monitoring the job}
\label{\detokenize{Section5:monitoring-the-job}}
Once the job has been submitted, the status and progress of the job can
be monitored from the ‘Control’ tab. This brings up a list of all of the
jobs that have recently been submitted as shown below.

\sphinxincludegraphics{{image61}.jpg}

\sphinxstylestrong{Figure 5.6:} Jobs listed in the ‘Control’ tab.

Clicking the following symbols at the end of the row allows information
on the simulation to be accessed among other things. The symbols are
described below:

\vspace{10mm}
\begin{savenotes}\sphinxattablestart
\centering
\begin{tabular}[t]{\X{15}{100}\X{85}{100}}
\sphinxstyletheadfamily 
\sphinxincludegraphics[scale=0.5]{{image72}.jpg}
&
Gives access to the log files from a given job.
\\
\sphinxincludegraphics[scale=0.5]{{image82}.jpg}
&
Takes you to the ‘Results’ tab to view the results from this model run.
\\
\sphinxincludegraphics[scale=0.5]{{image91}.jpg}
&
Reloads the setup of the model so that it can be reused or modified and resubmitted.
\\
\sphinxincludegraphics[scale=0.5]{{image101}.jpg}
&
Deletes the record of this model run.
\\
\end{tabular}
\par
\sphinxattableend\end{savenotes}


\chapter{A SPECFEM3D\_GLOBE simulation example}
\label{\detokenize{Section6::doc}}\label{\detokenize{Section6:a-specfem3d-globe-simulation-example}}

\section{Introduction}
\label{\detokenize{Section6:introduction}}
SPECFEM3D\_GLOBE is a spectral element code for simulating the seismic
wave-field on regional and global scales. As the solver accounts for the
curvature of the Earth is much better for large scale simulations.
Unlike SPECFEM3D\_Cartesian, SPECFEM3D\_GLOBE does not require a
pre-made mesh. Instead the mesh is produced for the simulation area by
an inbuilt meshing tool.

For global waveform simulations (i.e. simulations where the whole earth
is simulated) the user only defines the grid spacing for the whole earth
mesh. For regional simulations (where a segment of the globe is
simulated) the user defines the area to be modelled and the grid spacing
for that bespoke mesh. The mesh is then calculated before the simulation
is run.


\section{Setting the simulation area}
\label{\detokenize{Section6:setting-the-simulation-area}}
Once the user has selected the SPECFEM3D\_GLOBE solver from the drop
down ‘\sphinxstyleemphasis{solvers’} menu in the ‘\sphinxstyleemphasis{Solver’} tab the user will have the
option to select from two mesh types. The first of these is ‘\sphinxstyleemphasis{Globe’},
to run a global simulation. The second of these is ‘Bespoke’, which
allows a user defined area to be modelled.


\subsection{Global Simulation}
\label{\detokenize{Section6:global-simulation}}
If you select global simulation, then you are modelling the whole globe.
Therefore, all earthquakes and all seismic stations will be available
for modelling. The user can define the length of the simulation (by
varying the RECORD\_LENGTH parameter) and the resolution of the
simulation (by varying the mesh parameters) as described in section 6.4.
The ‘velocity model’ must then be set to default.


\subsection{Regional Simulation}
\label{\detokenize{Section6:regional-simulation}}
If you select a \sphinxstyleemphasis{‘Bespoke’} mesh you are able to define the specific
region you wish to simulate. This is defined by setting the centre of
the area to be modelled with the parameters
\sphinxstyleemphasis{‘CENTRE\_LONGITUDE\_IN\_DEGREES’} and
\sphinxstyleemphasis{‘CENTRE\_LATITUDE\_IN\_DEGREES’}. The width and height of the area to
be simulated are then defined by the \sphinxstyleemphasis{‘ANGULAR\_WIDTH\_XI\_IN\_DEGREES’}
and \sphinxstyleemphasis{‘ANGULAR\_WIDTH\_ETA\_IN\_DEGREES’} which describe the width of the
region at the closest point to the equator, and the height of the region
in degrees of longitude and latitude respectively. The approximate area
to be simulated is shown on the map to the left, allowing the area to be
refined and appropriate earthquakes and seismic stations to be selected
for simulation.

\sphinxincludegraphics{{image16}.png}

\sphinxstylestrong{Figure 6.1:} Example model setup for a global waveform simulation

\sphinxincludegraphics{{image23}.png}

\sphinxstylestrong{Figure 6.2:} Example of regional simulation for Southern European
and Mediterranean region.


\section{Selecting a velocity model}
\label{\detokenize{Section6:selecting-a-velocity-model}}
For both regional and global simulations a range of global 1D and 3D
velocity models can be used. These velocity models are defined within
SPECFEM3D\_GLOBE, and so are not defined in the \sphinxstyleemphasis{‘Velocity Model’} tab.
Instead the models are defined in the dropdown \sphinxstyleemphasis{‘MODEL’} menu in group 0
of the input parameters.

The input parameters for SPECFEM3D\_GLOBE are divided into 9 groups,
which are briefly described in Appendix 2.


\section{Defining the resolution \& mesh parameters}
\label{\detokenize{Section6:defining-the-resolution-mesh-parameters}}
In SPECFEM the frequency of seismic wave that can be accurately
simulated depends on both the grid spacing (\sphinxstyleemphasis{DH}) of the mesh of points
the wave-field is calculated for, and the time step (\sphinxstyleemphasis{DT}). Unlike
SPECFEM3D\_Cartesian, (where \sphinxstyleemphasis{DT} is manually defined, and should be set
to an appropriate value for the mesh and velocity model that are
selected), SPECFEM3D\_GLOBE calculates the \sphinxstyleemphasis{DT} that is needed, based on
the grid spacing of the mesh. The user therefore defines the highest
frequency (or shortest wavelength) that the setup can accurately
simulate by setting the mesh parameters.

For the Global case the user must first define how many ‘chunks’ the
globe should be subdivided into, the default on the platform being 6.
For regional simulations the simulation area must be defined as a single
chunk. The resolution of the simulation is then prescribed by the values
‘NEX\_XI’ and ‘NEX\_ETA’, which correspond to the number of elements at
the surface of the model space for the first chunk in the side of length
XI and ETA respectively. SPECFEM3D\_GLOBE requires that the value of
NEX\_XI must be a multiple of 16, and be 8 times a multiple of
NPROC\_XI. In turn the value of NEX\_ETA must be a multiple of 16, and
be 8 times a multiple of NPROC\_ETA. To summarise;
\begin{equation*}
\begin{split}\text{NE}X_{\text{XI}} = \ c \times 8\left( \text{NPRO}C_{\text{XI}} \right)\end{split}
\end{equation*}\begin{equation*}
\begin{split}\text{NE}X_{\text{ETA}} = \ c \times 8\left( \text{NPRO}C_{\text{ETA}} \right)\end{split}
\end{equation*}
Where \((\text{NPO}C_{\text{ETA}}*c)\) and
\((NPOC_{\text{XI}}*c)\) are even and greater than 2.

The shortest period resolved by the simulation can then be approximated
by the following equation;
\begin{equation*}
\begin{split}
\text{shortest period }\left( s \right) \simeq \left( \frac{256}{\text{NE}X_{\text{XI}}} \right) \times \left( \frac{\text{ANGULAR WIDTH XI IN DEGREES}}{90}\  \right) \times 17
\end{split}
\end{equation*}
For regional simulations the areas you have defined at the Earth’s
surface will be modelled. The depth of the simulation area is
automatically defined, and the wave filed is simulated down to the inner
core boundary in regional simulations. This means that certain seismic
phases that travel through the core will not be seen. The VERCE platform
automatically sets absorbing boundary conditions for simulation areas.
However, as these boundary conditions are not perfect care should be
taken if using receivers or in particular sources that are close to the
limits of the region simulated.

Once you have defined the area you wish to model the earthquake sources
and receivers can be defined in the ‘\sphinxstyleemphasis{Earthquakes} ’ and
‘\sphinxstyleemphasis{Stations’} tabs.


\section{Selecting an Earthquake}
\label{\detokenize{Section6:selecting-an-earthquake}}
For global and regional simulations the only earthquake catalogue that
is currently supported is the gCMT catalogue. Other earthquake
mechanisms can be uploaded is gCMT format using the ‘\sphinxstyleemphasis{file}’ tab.
The format needed is shown in figure 6.5.

If you are using the gCMT catalogue supported in the portal, you can
search for events of certain magnitudes and dates using the search
parameter boxes, and the available earthquakes are seen inside the mesh
area. This is shown for global simulation (figure 6.3) and a regional
simulation (figure 6.4) below. The relevant earthquake can then be
selected from either the map (by clicking on the red dot) or from the
list to the right.

\sphinxincludegraphics{{image33}.png}

\sphinxstylestrong{Figure 6.3:} earthquake search for a global simulation.

\sphinxincludegraphics{{image42}.png}

\sphinxstylestrong{Figure 6.3:} earthquake search for a regional simulation for
Europe.

\sphinxincludegraphics{{image52}.jpg}

\sphinxstylestrong{Figure 6.4:} Format of a bespoke CMT solution for upload to the
portal. (Image from the SPECFEM3D Globe manual).


\section{Selecting stations}
\label{\detokenize{Section6:selecting-stations}}
Stations can then be selected as for Cartesian simulations. For
SPECFEM3D\_GLOBE the only source of station locations that is built in
is the IRIS catalogue. If you wish to add further stations, station
locations can be manually uploaded in the format shown in figure 6.5.

Example station searches for a regional simulation of Europe and for a
global simulation are shown in figures 6.6 and 6.7 respectively. In the
global simulations particularly, it can be useful to specify the
specific networks needed by specifying the network code in the
‘\sphinxstyleemphasis{Networks’} drop-down box. Stations from multiple networks can be
search by separating the network codes with a comma only (no space) e.g.
(IU,II). If all stations are selected the station searching and data
parsing will be very slow, and it is unlikely that this volume of data
will be useful. So please take the time to search for useful stations
carefully.

\sphinxincludegraphics{{image62}.jpg}

\sphinxstylestrong{Figure 6.5:} Format for manual station location upload (Image from
the SPECFEM3D\_GLOBE manual)

\sphinxincludegraphics{{image72}.png}

\sphinxstylestrong{Figure 6.6:} Searching for stations in a regional mesh.

\sphinxincludegraphics{{image82}.png}

\sphinxstylestrong{Figure 6.7:} Searching for stations in a global mesh.


\section{Submitting and monitoring the simulation}
\label{\detokenize{Section6:submitting-and-monitoring-the-simulation}}
Once the simulation has been setup, the relevant earthquake source(s)
defined, and the required stations selected the simulation can be
submitted to the supercomputer. The workflow is selected from the top
drop-down box, with different workflows corresponding to different HPC
resources. The name and description boxes allow you to document exactly
what this simulation is for.

Once the model has been submitted the progress of the simulation can be
monitored on the ‘results’ tab. As the processing is done on a HPC
machine, the simulation may not run immediately, as these large jobs are
managed in a queue system.

\sphinxincludegraphics{{image92}.png}

\sphinxstylestrong{Figure 6.8:} Submitting a job, for a regional simulation of Europe
using SPECFEM3D\_GLOBE.

\sphinxincludegraphics{{image102}.png}

\sphinxstylestrong{Figure 6.9:} Monitoring the progress of submitted job through the
‘results’ tab.


\section{Outputs from regional \& global simulations}
\label{\detokenize{Section6:outputs-from-regional-global-simulations}}
The results of these simulations can then be accessed through the
results tabs as described in the next chapter. Example outputs include
waveforms (figure 6.10), a .kmz file that can be used to view the
waveforms in geographical context (figure 6.11), as well as global and
regional snapshots and movie animations (e.g. figure 6.12).

\sphinxincludegraphics{{image112}.png}

\sphinxstylestrong{Figure 6.10:} Waveform produced by regional simulation.

\sphinxincludegraphics{{image121}.png}

\sphinxstylestrong{Figure 6.11:} Waveforms in geographical context using KMZ file
viewed in Goole Earth.

\sphinxincludegraphics{{image131}.png}

\sphinxstylestrong{Figure 6.12:} Snapshot from a regional simulation of Europe, using
regional settings in SPECFEM3D Globe.


\chapter{Processing and Accessing the Results}
\label{\detokenize{Section7::doc}}\label{\detokenize{Section7:processing-and-accessing-the-results}}

\section{Outputs of the forward simulations}
\label{\detokenize{Section7:outputs-of-the-forward-simulations}}
Once your job has run, you will be able to access the output of the
simulation in three ways:
\begin{enumerate}
\item {} 
Clicking on the ‘\sphinxstyleemphasis{Control’} tab in the ‘\sphinxstyleemphasis{Simulation’} section of
the ‘\sphinxstyleemphasis{Forward} \sphinxstyleemphasis{Modelling’} panel (see section 5.7):

It shows a list of the simulation jobs that have been launched with,
among others, the corresponding ‘\sphinxstyleemphasis{Name}’, ‘\sphinxstyleemphasis{Description’},
‘\sphinxstyleemphasis{Status’} and ‘\sphinxstyleemphasis{Date’} of the run; if you do not see the event you
have just run then click ‘\sphinxstyleemphasis{Refresh list}’ to load this in. By
clicking on the blue-eye icon next to each run you will be redirected to
the ‘\sphinxstyleemphasis{Results’} section showing the selected run and all its outputs
(in the far left of the opened panel) as described in the next point.

\item {} 
Directly checking under the ‘\sphinxstyleemphasis{Results’} section:

The simulation results can be searched using the ‘\sphinxstyleemphasis{Open Run’} button
which is on the top left. This enables you to search for runs that for
instance involve earthquakes in a range of magnitudes (as shown in
Figure 7.1), or a range of depths, latitudes, longitudes, etc. For a
full list of the parameters for which it is possible to search for,
please see the ‘\sphinxstyleemphasis{Terms’} drop down menu. You can then select the
simulation of interest from the list of previous runs satisfying the
searching criteria that appears on the screen after clicking the
‘\sphinxstyleemphasis{Search’} button. As in case 1), the far left of the panel will then
show the outputs of this simulation.

\item {} 
Using the ‘\sphinxstyleemphasis{iRODS}’ panel: see section 7.4.

\sphinxincludegraphics{{image17}.png}

\sphinxstylestrong{Figure 7.1:} Example of searching simulations involving events in a
selected magnitude range.

\end{enumerate}

Focusing now on points 1) and 2), selecting an output from the left hand
part of the ‘\sphinxstyleemphasis{Results}’ panel will bring up a provenance diagram in
the top right window of the ‘\sphinxstyleemphasis{Results’} section marked as ‘\sphinxstyleemphasis{Data
Dependency Graph’}. If this is done for instance for the .kml file that
is output, all of the constituent inputs needed for this result are
shown by a dark blue circle bounded in yellow, and the outputs are shown
by a simple dark blue circle.

The bottom right window, marked as \sphinxstyleemphasis{‘Data products’}, gives further
details of the output file that is selected and lists possible errors
occurred during the production of this output. To open the file
concerned, click the blue link marked ‘\sphinxstyleemphasis{Open’} or the one marked
‘\sphinxstyleemphasis{Download’} appearing in the dialog window. Moreover, with the button
\sphinxstyleemphasis{‘Produce Download Script’} you can get a piece of code to download the
selected output via gsissh terminal.

Using the ‘\sphinxstyleemphasis{Search’} button in the ‘\sphinxstyleemphasis{Data} \sphinxstyleemphasis{products’} section of
the ‘\sphinxstyleemphasis{Results’} panel it is also possible to search for all the output
files of a specific mime-type (e.g., png, kmz, etc.) for a specific
simulation.

Finally, you can also visualise the input files for the selected
simulation, such as the quakeML file, which contains information about
the source or sources that are input into the model. This is done using
the ‘\sphinxstyleemphasis{View Inputs’} button on the top left of the ‘\sphinxstyleemphasis{Results’} panel.


\subsection{Waveform outputs}
\label{\detokenize{Section7:waveform-outputs}}
The primary outputs of any seismic simulation are the recorded
waveforms. These can be viewed most simply as a .png file as shown in
figure 7.2. To access these figures, after selecting a given simulation
(as explained above), use the ‘\sphinxstyleemphasis{Search’} button in the ‘\sphinxstyleemphasis{Data}
\sphinxstyleemphasis{products’} window (of the ‘\sphinxstyleemphasis{Results’} panel) to search for
‘image/png’ type of files. Otherwise, waveforms can be downloaded as
seed files searching and downloading ‘application/octet-stream’
mime-type files.
\begin{quote}

\sphinxincludegraphics{{image24}.png}

\sphinxstylestrong{Figure 7.2:} Example of waveform output.
\end{quote}

The simulation code produces one file for each of the three components
of a seismic station and, depending on the Par\_file set up (see
Appendix 1), the seismograms can be in displacement, velocity or
acceleration, or all of them.

The three components of different seismometers that are output can also
be viewed in a more interactive form, by downloading the *.kmz file
that is automatically output from the simulation run and viewing it in
Google Earth as shown in Figure 7.3. This kmz file can be downloaded by
searching for a mime-type ‘\sphinxstyleemphasis{application/vnd.google-earth.kmz’} in the
‘\sphinxstyleemphasis{Search’} section of ‘\sphinxstyleemphasis{Data} \sphinxstyleemphasis{products’}.
\begin{quote}

\sphinxincludegraphics{{image34}.png}

\sphinxstylestrong{Figure 7.3:} Three components of a synthetic seismogram produced
for an earthquake in Central Italy, observed on the interactive
Google Earth tool. The seismograms are shown in displacement,
velocity and acceleration.
\end{quote}


\subsection{Animation outputs}
\label{\detokenize{Section7:animation-outputs}}
The VERCE platform can also be used to produce animations or movies of
the waveforms propagating out from the simulated earthquake event. These
animations can be projected onto the Earth’s surface as in the snapshot
example of Figure 7.4, or can show the propagation over all the external
faces of the mesh (i.e., topography+vertical edges+bottom) depending on
the Par\_file set up as described in Appendix 1.

The movie file *.mp4 is automatically output from the portal and can be
downloaded by searching for a mime-type
‘\sphinxstyleemphasis{application/vnd.google-earth.kmz}’ in the ‘\sphinxstyleemphasis{Search’} section of
‘\sphinxstyleemphasis{Data} \sphinxstyleemphasis{products’} (in the ‘\sphinxstyleemphasis{Results’} panel). The animations shown
at the top of the movie file and the one on the bottom left (Figure 7.4)
represent the three components of the waveform velocity propagation,
while the animation at the bottom right is the instantaneous peak ground
velocity, i.e. a map of the maximum ground velocity for each time step
of the simulation.

\sphinxincludegraphics{{image43}.png}

\sphinxstylestrong{Figure 7.4:} Snapshot of the movie for an earthquake in Central
Italy produced from the VERCE platform using a regional simulation in
SPECFEM3D\_Globe.


\subsection{Other outputs}
\label{\detokenize{Section7:other-outputs}}
The portal outputs can also be processed externally with your own
routines to produce, for example, ground motion maps as shown for an
event in Northern Italy in Figure 7.5. In particular, this kind of maps
can be obtained by processing a binary file called \sphinxstyleemphasis{shakingdata}
produced in output by SPECFEM3D\_Cartesian if in the Par\_file the user
sets the flag

CREATE\_SHAKEMAP = .true.

\sphinxincludegraphics{{image53}.png}

\sphinxstylestrong{Figure 7.5:} A ground shaking map produced with SPECFEM3D\_Cartesian
for an event occurring in Northern Italy.

One of the main changes that will be introduced in the next release of
the portal (autumn/winter 2017) is the automatic production, through a
dedicated workflow in the portal, of such ground motion maps, i.e. PGD,
PGV and PGA maps. A single *.png picture containing the three maps will
be generated from the automatic processing of the \sphinxstyleemphasis{shakingdata} file.


\section{Downloading observed data}
\label{\detokenize{Section7:downloading-observed-data}}
One of the main goals of a seismological analysis is the comparison of
simulated results with observed data. Thus, after running a waveform
simulation, the portal allows users to download from European data
archives the recorded seismograms corresponding to the simulated
earthquake.

The section marked as ‘\sphinxstyleemphasis{Download’} under the ‘\sphinxstyleemphasis{Forward Modelling’}
panel is dedicated to this task and it is composed by the following
sub-sections that guide the users to launch a data -download job.
\begin{itemize}
\item {} 
\sphinxstyleemphasis{’Setup’} tab:

A list of all the simulations that have been run is shown in this
window. By selecting one of them, the information on the corresponding
earthquake (source location, origin time) and run (NSTEP, DT) are
automatically passed to the portal thanks to the metadata stored along
with each simulation job.

\item {} 
\sphinxstyleemphasis{‘Submit’} tab:

In the upper window you see the input parameters for the download job
based on the selection in the previous tab; in the lower window you can
setup the submission parameters, in particular the specific workflow and
the number of cores of the HPC resource to run the data-download job
(see the example in Figure 7.6). Click on ‘\sphinxstyleemphasis{Submit’} on the lower
right to launch the job.

\item {} 
\sphinxstyleemphasis{‘Control’} tab:

As for the case of simulation jobs in section 7.1, this window shows a
list of the download jobs that have been launched and their
corresponding status, among other information. The \sphinxstyleemphasis{‘blue-eye’} icon
next to each run links you again to the main ‘\sphinxstyleemphasis{Results’} page where
this time you visualize the outputs of the selected data-download job
and the provenance graphs.

Otherwise, the outputs can be accessed by searching for the specific
download job directly from the ‘\sphinxstyleemphasis{Results’} panel (see section 7.1) or
from the ‘\sphinxstyleemphasis{iRODS}’ panel (see section 7.4).

\end{itemize}

The output files of a data-download job are both seed and png files of
the recorded traces downloaded from European archives. Use the
‘\sphinxstyleemphasis{Search’} button in the \sphinxstyleemphasis{‘Data products’} window of the ‘\sphinxstyleemphasis{Results’}
tab to search for mime-type ‘\sphinxstyleemphasis{application/octet-stream’} or
‘\sphinxstyleemphasis{image/png’}, respectively.

\sphinxincludegraphics{{image63}.png}

\sphinxstylestrong{Figure 7.6:} Example of submission settings for a data-download
job.


\section{Waveform processing}
\label{\detokenize{Section7:waveform-processing}}
Once both a simulation job and a download job have been run, users can
exploit another feature of the portal to pre-process observed and
synthetic waveforms in order to prepare them for comparison analyses.

In the ‘\sphinxstyleemphasis{Forward Modelling’} panel, by clicking on the
‘\sphinxstyleemphasis{Processing’} tab you should go through the following sub-sections to
set up and launch a processing job.
\begin{itemize}
\item {} 
\sphinxstyleemphasis{‘Data Setup’} tab:

The window on the top left shows a list of the waveform simulations that
have been run, while the window on the top right shows a list of the
data-download jobs that have been run. By selecting a simulation and a
download run, a list of the seismic stations involved in the two jobs
appears in the window below and the portal automatically highlights the
common stations, i.e. those for which both data and synthetics are
stored in its database. Use the checkbox on the left of each station to
select those for which you are interested in processing both simulated
and recorded seismograms in order to then compare them.

\item {} 
‘\sphinxstyleemphasis{Processing Setup’} tab:

Here you can built-up a customized pipeline of processing analyses to be
applied to the recorded and simulated waveforms selected in the previous
panel. A list of the most common seismological processing functions
(here called \sphinxstyleemphasis{Processing Elements}, and abbreviated to \sphinxstyleemphasis{PEs}) is
reported in the far left window. Drag the selected PEs into the top
right window following the order of the operations you want to apply on
the seismograms.

For each PE of the pipeline you can choose to apply the operation on
both data and synthetics or on just one of them using the checkboxes
‘\sphinxstyleemphasis{raw’} and ‘\sphinxstyleemphasis{synt’}. As an example, usually the operation of
removing the instrument response (represented by the PE
\sphinxstyleemphasis{remove\_response}) is applied only on the raw data, while on the
synthetics one applies a pre-filtering function (\sphinxstyleemphasis{pre\_filter} PE) to
replicate the bandpass filter done by \sphinxstyleemphasis{remove\_response} function on the
data but without any deconvolution (see Figure 7.7). Moreover, the
checkbox ‘\sphinxstyleemphasis{visu’} allows you to produce a png file showing the result
of the specific analysis, while the checkbox ‘\sphinxstyleemphasis{store’} allows to also
store the processed seismogram as a seed file.

For each PE you can also modify the corresponding parameters. Clicking
on the row of a given PE in the ‘\sphinxstyleemphasis{PE Workflow’} window, the
corresponding parameters appear in the window below and you can set up,
for example, the type of de-trend, the type of taper and its percentage,
the limit frequencies for the selected filter.

Finally, on the top of the ‘\sphinxstyleemphasis{PE Workflow’} window a drop down menu
allows to select if the output of the processing will be in
displacement, velocity or acceleration, and with a checkbox you can
decide to rotate the seismic traces from NS and EW to radial and
vertical components.

\item {} 
\sphinxstyleemphasis{‘Submit’} tab:

As in the case of section 7.2, this tab shows a summary of the set up
for the processing job, in particular the list of stations to which the
processing will be applied and a list of the processing operations that
compose the custom pipeline. Then, you can setup the submission
parameters in the lower window and launch the processing job by clicking
on ‘Submit’ on the lower right.

\item {} 
\sphinxstyleemphasis{‘Control’} tab:

As previously, the window shows a list of the processing jobs that have
been launched and the blue-eye icon links to the main ‘\sphinxstyleemphasis{Results}’
page where the outputs of the selected job can be explored together with
the provenance graphs. The outputs can be also accessed by searching for
the specific processing job directly from the ‘\sphinxstyleemphasis{Results}’ panel (see
section 7.1) or from the ‘\sphinxstyleemphasis{iRODS}’ panel (see section 7.4).

\end{itemize}

The output products of a processing job can be png files of the
processed traces, if the option ‘\sphinxstyleemphasis{visu’} is checked, or/and seed files
containing the processed seismograms, if the option ‘\sphinxstyleemphasis{store}’ is on
(see above). Use the \sphinxstyleemphasis{‘Search’} button in the ‘\sphinxstyleemphasis{Data products’} window
of the \sphinxstyleemphasis{‘Results’} tab to search for mime-type ‘\sphinxstyleemphasis{image/png’} or
‘\sphinxstyleemphasis{application/octet-stream’}, respectively.

\sphinxincludegraphics{{image73}.png}

\sphinxstylestrong{Figure 7.7:} Example of custom pipeline of processing functions to
be applied on observed and synthetic seismograms.


\section{Misfit calculation}
\label{\detokenize{Section7:misfit-calculation}}
After simulating the seismic wave field, downloading the raw seismic
data and pre-processing both, the VERCE portal allows users to compare
synthetic and recorded seismograms and to quantitatively assess the
goodness of fit. Evaluating this fit is essential to approach the
inverse problem in seismology and there are numerous algorithms and
procedures to accomplish this task. In the portal we have so far
implemented two different established techniques for misfit calculation
and a third option that combines the two:
\begin{enumerate}
\item {} 
\sphinxstylestrong{PYFLEX}

This is a python port (L. Krisher; \sphinxurl{http://krischer.github.io/pyflex}) of
the fortran code FLEXWIN (Maggi et al., 2009;
\sphinxurl{http://geodynamics.org/cig/software/flexwin}). Considering full observed
and synthetic traces, the code selects a set of time-windows suitable
for waveform comparison based on given input parameters and estimates
cross-correlation, time-shift and amplitude ratio within each window.
The code allows for an automated selection of the windows handling large
data volumes and also complex 3D simulated waveforms, hence it is
particularly useful for iterative tomographic inversions (Maggi et al.,
2009). For a complete description of the method and the parameters see
the manual of PYFLEX (or FLEXWIN).

In the portal this option corresponds to ‘\sphinxstyleemphasis{misfit\_type = pyflex’}.

\item {} 
\sphinxstylestrong{Kristeková’s misfit method}

This is a python code based on the method developed by Kristeková et al.
(2006) and Kristeková et al. (2009). The method compares observed and
synthetic full-waveforms and allows the following time-frequency (TF)
misfit criteria to be estimated:
\begin{itemize}
\item {} 
time-frequency envelope misfit (TFEM)

\item {} 
time-frequency phase misfit (TFPM)

\item {} 
time envelope misfit (TEM)

\item {} 
time phase misfit (TPM)

\item {} 
frequency envelope misfit (FEM)

\item {} 
frequency phase misfit (FPM)

\item {} 
envelope misfit (EM)

\item {} 
phase misfit (PM)

\end{itemize}

This method allows for comparing arbitrary time signals in their entire
TF complexity, thus providing a detailed TF anatomy of the disagreement
between two full signals from the point of view of both envelope and
phase (Kristeková et al., 2009). For a complete description of the
method see Kristeková et al. (2006) and Kristeková et al. (2009).

In the portal this option corresponds to ‘\sphinxstyleemphasis{misfit\_type =
time\_frequency’}.

\item {} 
\sphinxstylestrong{PYFLEX + Kristeková’s misfit method}

In this case the time windows are selected using the code PYFLEX and
then the time-frequency misfit criteria are estimated on this windows
using Kristekova’s method.

In the portal this corresponds to ‘\sphinxstyleemphasis{misfit\_type =
pyflex\_and\_time\_frequency’}.

\end{enumerate}

The section of the VERCE portal for misfit calculation is accessible
through the ‘\sphinxstyleemphasis{Misfit}’ tab in the ‘\sphinxstyleemphasis{Forward Modelling}’ panel
and it consists of the following sub-sections that allows for the set-up
of a misfit job.
\begin{itemize}
\item {} 
\sphinxstyleemphasis{‘Setup’} tab:

\end{itemize}
\begin{quote}

Select one of the processing job that have been run and that are listed
in the upper window of this panel. Then select the ‘\sphinxstyleemphasis{Misfit type}’
from the drop-down menu considering that
\end{quote}
\begin{itemize}
\item {} 
‘\sphinxstyleemphasis{pyflex’} corresponds to option 1 above

\item {} 
‘\sphinxstyleemphasis{time}\_\sphinxstyleemphasis{frequency’} corresponds to option 2 above

\item {} 
‘\sphinxstyleemphasis{pyflex\_and\_time\_frequency’} corresponds to option 3 above

\end{itemize}
\begin{quote}

For each misfit procedure the lower window of the panel shows the
corresponding parameters that should be set up by the user. In
particular, for option 1 \textendash{} ‘PYFLEX’ the tuning parameters control
the window selection and are fully described in the manual of the
code; for option 2 \textendash{} ‘Kristeková’s misfit method’ the main
parameters are the minimum and maximum period at which the waveforms
have been filtered; option 3 \textendash{} ‘PYFLEX + Kristeková’s misfit method‘
contains all the parameters of the two previous options. (See Figure 7.8).
\end{quote}
\begin{itemize}
\item {} 
‘\sphinxstyleemphasis{Submit’} tab:

A summary of the chosen misfit method and set up parameters is shown in the upper window of this section. Then, you can setup the submission
parameters in the lower window and launch the misfit job by clicking on ‘\sphinxstyleemphasis{Submit}’ on the lower right.

\item {} 
‘\sphinxstyleemphasis{Control}’ tab:

As always, the window shows a list of the misfit jobs that have been
launched and the blue-eye icon links to the main ‘\sphinxstyleemphasis{Results}’ page
where the outputs of the selected job can be explored together with the
provenance graphs. The outputs can be also accessed by searching for the
specific misfit job directly from the ‘\sphinxstyleemphasis{Results}’ panel (see section
7.1) or from the ‘\sphinxstyleemphasis{iRODS}’ panel (see section 7.4).

\end{itemize}

The output products of a misfit job are png files showing the waveform
comparison for each component of each selected seismic station. The
figures are different depending on the misfit option chosen in the
‘\sphinxstyleemphasis{Setup}’ tab (see examples in Figures 7.9 and 7.10). To access
these output files use the ‘\sphinxstyleemphasis{Search}’ button in the ‘\sphinxstyleemphasis{Data
products}’ window of the ‘\sphinxstyleemphasis{Results}’ tab searching for mime-type
‘\sphinxstyleemphasis{image/png}’.

\sphinxincludegraphics{{image83}.png}

\sphinxstylestrong{Figure 7.8:} Example of the set-up of a misfit job using in
combination PYFLEX and Kristeková’s method.

\sphinxincludegraphics{{image93}.png}

\sphinxstylestrong{Figure 7.9:} Example of an output file produced by calculating the
misfit using PYFLEX. For each component of each station the figure shows
the observed data in black, the synthetic trace in red and the
short-term average/long-term average ratio in blue; the windows selected
by the code are highlighted.

\sphinxincludegraphics{{image103}.png}

\sphinxstylestrong{Figure 7.10:} Example of an output file produced by calculating the
misfit using Kristeková’s method. For each component of each station the
figure shows the observed data in black, the synthetic trace in red;
moreover, it shows the Kristeková’s misfi criteria TFEM, TFPM, TEM, TPM,
FEM, FPM, EM, PM (see the text for details).


\section{Accessing the results through iRODS}
\label{\detokenize{Section7:accessing-the-results-through-irods}}
After any of the jobs described above (simulation, download, processing
and misfit) is finished, the output products are shipped from the HPC
resource, that performed the calculation, to the store repository of the
VERCE portal. This storage is managed through iRODS that provides a
repository shared among the federated nodes of the VERCE organisation.
How to create an account in iRODS is described in section 3 of this
manual, and after that you can log in selecting the ‘\sphinxstyleemphasis{IRODS}’ panel
in the portal main menu (see Figure 7.11).

\sphinxincludegraphics{{image113}.png}

\sphinxstylestrong{Figure 7.11:} Screenshot of the ‘IRODS’ panel in the VERCE portal

Entering iRODS gives you access to your personal folder where you can
navigate the results of all your completed jobs (of any type) as
anticipated above. The run results are organised in trees of subfolders
with the main directory called using the ‘\sphinxstyleemphasis{Name}’ you have chosen
for your job (see section 5.6), as shown for example in Figure 7.12.

\sphinxincludegraphics{{image122}.png}

\sphinxstylestrong{Figure 7.12:} Example of the iRODS subfolder structure containing
the results of the jobs.

A given job can be selected by double-clicking on the relative folder
and, navigating the subdirectories, you can access all the same input
and output files of each job that have been described in the above
sections.

It is very important that in order to visualise or download any output
or input data from the portal, both via the ‘\sphinxstyleemphasis{Results}’ tab or the
‘\sphinxstyleemphasis{iRODS}’ tab, you always need to firstly log in into the
‘\sphinxstyleemphasis{iRODS}’ panel (Figure 7.11) because the storage database is
accessible only to authenticated users.


\chapter{Running SPECFEM3D\_Cartesian simulations using your own data}
\label{\detokenize{Section8:running-specfem3d-cartesian-simulations-using-your-own-data}}\label{\detokenize{Section8::doc}}
So far we have covered how to run a relatively simple simulation using
data that are already loaded into the portal. While we hope to keep
increasing the areas where meshes and models are currently available, it
is also essential that users can upload their own velocity models,
meshes, earthquake catalogues and station locations. In this section we
will give a brief overview of what is needed for each of these inputs,
and how they are submitted to the portal.


\section{Creating your own velocity model}
\label{\detokenize{Section8:creating-your-own-velocity-model}}
All of the velocity models currently loaded into the VERCE portal are
based on regional travel time tomography models of the concerned area.
It would be possible to construct an input velocity model based on a
surface wave tomography or noise tomography, or even from a velocity
model produced from active seismic techniques based on refraction or
reflection seismic surveys. If you are interested in running a full
waveform simulation in a new area, you will therefore acquire a
published velocity model, or use a preliminary model from your own work
or collaborations.

The new velocity model should be defined in a 3D grid of points and the
corresponding input file for the portal should be formatted as shown
below (Figure 8.1) and saved as a text file, usually called tomography
file.

The velocity model you submit should be based on your tomography, and
should deviate back to the regional or global 1D starting model at the
edges of the 3D volume. The tomography file should be formatted as shown
in the below (Figure 7.1), and saved as a text file.

The grid spacing defined on the second row of the file depends upon the
frequency of seismic wave you intend to simulate, as described in
section 8.2. Once this file has been created, the velocity model can be
uploaded along with the corresponding mesh as described in section 8.3.

The variables input into the velocity model text file are defined as
follows.

\sphinxincludegraphics{{image14}.jpg}

\sphinxstylestrong{Figure 8.1:} Format for an input velocity model. Image re-produced
from the SPECFEM manual.

\sphinxstylestrong{ORIG\_X, ORIG\_Y,} \& \sphinxstylestrong{ORIG\_Z}: are the coordinates of the initial
grid points in the tomographic model in the x, y and z directions
respectively.

\sphinxstylestrong{END\_X, END\_Y,} \& \sphinxstylestrong{END\_Z:} are the coordinates of the final grid
points in the tomographic model in the x, y and z directions
respectively.

\sphinxstylestrong{SPACING\_X, SPACING\_Y,} \& \sphinxstylestrong{SPACING\_Z}: describe the spacing
between points of the tomography file in the x, y and z directions
respectively.

\sphinxstylestrong{NX, NY,} \& \sphinxstylestrong{NZ}: describe the number of grid points in the x, y and
z directions respectively.

\sphinxstylestrong{VP\_MIN} \& \sphinxstylestrong{VP\_MAX:} describe respectively the minimum and maximum
P-wave velocity of the input file in \sphinxstyleemphasis{ms:sup:{}`-1{}`}.

\sphinxstylestrong{VS\_MIN} \& \sphinxstylestrong{VS\_MAX}: describe respectively the minimum and maximum
S-wave velocity of the input file in \sphinxstyleemphasis{ms:sup:{}`-1{}`}.

\sphinxstylestrong{RHO\_MIN} \& \sphinxstylestrong{RHO\_MAX}: describe respectively the minimum and
maximum density of the input file in \sphinxstyleemphasis{kg/m:sup:{}`3{}`}.


\section{Creating a bespoke mesh for your area}
\label{\detokenize{Section8:creating-a-bespoke-mesh-for-your-area}}
Creating a mesh is the most complicated step in setting up a simulation
in a new area. The mesh must be created so that it can account for the
frequency of seismic waves at the seismic velocities that are found in
the velocity model defined above.


\subsection{Meshing parameters}
\label{\detokenize{Section8:meshing-parameters}}
The spacing of the grid (\({\Delta}h\)) depends upon the minimum seismic
velocity in the wave speed model (\(v_{\min}\)) and the frequency
(\(1/To,\)) to which you wish to resolve the wavefield in your
simulations as shown in the equation below (Komatitsch et al., 2005):
\begin{equation}\label{equation:Section8:eq1}
\begin{split}{\Delta}h = \ v_{\min}\text{\ T}_{o}\ \frac{n + 1}{f(n)}\end{split}
\end{equation}
where \(\text{To}\) is the shortest period that can be resolved, and
\(n\) is the degree of polynomials used to represent the wave field
in the spectral element method. Seismic velocity usually increases with
depth, thus, in order to have the same resolution everywhere in the
model, element size should increase.

The time integration algorithm used by SEM to solve the seismic wave
equation is conditionally stable, i.e. there exists an upper limit to
the value of the time step over which the calculations become unstable.
The stability condition, namely the Courant stability condition, is
given by (Komatitsch et al., 2005):
\begin{equation}\label{equation:Section8:eq2}
\begin{split}{\Delta}t \leq C_{\max}\ \left( \frac{h}{v} \right)_{\min}\end{split}
\end{equation}
where \({(\ \frac{{\Delta}h}{v}\ )}_{\min}\) denotes the minimum ratio
between the grid spacing and the P-wave velocity, and \(C_{\max}\)
is the highest possible value of the Courant number. Based on equation
\eqref{equation:Section8:eq1}, equation \eqref{equation:Section8:eq2} can be written as
\begin{equation}\label{equation:Section8:eq3}
\begin{split}{\Delta}t \leq C_{\max}\ \frac{v_{\min}}{v_{\max}}\ T_{o}\ \frac{n + 1}{f(n)}\end{split}
\end{equation}
Finally, the mesh should also account for the topography of the Earth’s
surface, or bathymetry if the modelled area includes oceanic areas.

\sphinxincludegraphics{{image25}.png}

\sphinxstylestrong{Figure 8.2:} Example of a hexahedral mesh built using GEOCUBIT. On
the left of the figure we can see that the grid spacing increases with
depth as the wave speed increases. The top surface of the mesh
represents the topography of the area to be modelled.


\subsection{Meshing software}
\label{\detokenize{Section8:meshing-software}}
Meshes that can be used with SPECFEM3D\_Cartesian and within the portal
can be produced using CUBIT/ TRELIS software. Unfortunately the
CUBIT/TRELIS software is not free, although a 30 day trial licence can
be downloaded. Full details of the commercial software can be found at
the following link.

\sphinxurl{http://www.csimsoft.com/}

This software is then used in conjunction with the free python based
GeoCubit software developed at the INGV:

\sphinxurl{https://github.com/geodynamics/specfem3d/tree/devel/CUBIT\_GEOCUBIT}

A full description of how to create a mesh for SPECFEM3D\_Cartesian
implemented into the VERCE platform can be found here:

\sphinxurl{http://verce.eu/Training/UseVERCE/2015-7-VERCE-hexmeshing101.pdf}


\section{Submitting a mesh and a velocity model}
\label{\detokenize{Section8:submitting-a-mesh-and-a-velocity-model}}
Once you have produced your mesh, you can take the ten mesh files listed
in figure 8.3.1 and put them into a folder named
‘\sphinxstyleemphasis{mesh\_MySimulationArea’}, where ‘\sphinxstyleemphasis{MySimulationArea’} is the name
of the area you are studying. Then, zip this folder in a single zip file
named ‘\sphinxstyleemphasis{mesh\_ MySimulationArea.zip’}.

Do the same with the tomography file formatted as outlined in section
8.1. The tomography file can have whatever name you want but must be put
in a folder named ‘\sphinxstyleemphasis{velocity\_ MySimulationArea’}. Finally, zip this
folder in a single zip file named ‘\sphinxstyleemphasis{velocity\_ MySimulationArea.zip’,
as shown in Figure 8.3.1}.

\sphinxincludegraphics{{image35}.png}

\sphinxstylestrong{Figure 8.3.1:} Creating the zip files needed to upload a new model to
the VERCE platform. The left hand side shows the 10 mesh files that need
to be included and uploaded. The right hand side shows the single
velocity file that needs to be uploaded, along with the naming
conventions for these files.

Once these zip files have been created, they can be uploaded by clicking
the link labelled ‘\sphinxstylestrong{*Click here to submit a new mesh and velocity
model’*} in the solver tab of the forward modelling page. This will
bring up the parameter form shown in figure 8.3.2. The zipped mesh and
velocity model files can then be uploaded to the portal from the local
machine. The limits of the mesh area in latitude and longitude should
also be input in the ‘mesh bounds’ section of the pop-up window. Finally
click ‘Submit’ for the mesh and velocity model to be uploaded to the
portal. The meshes and models are manually validated before they are
made available to you and to all the users, so it can take several days
for the mesh and model to be uploaded and ready to use.

It is of course possible to upload meshes and models that only you or
your group of users can use. In this case please specify it in the note
box at the bottom of the parameter form.

\sphinxincludegraphics{{image44}.png}

\sphinxstylestrong{Figure 8.3.2:} The parameter form for inputting a new mesh and model
into the VERCE platform.


\section{Submitting a new earthquake catalogue}
\label{\detokenize{Section8:submitting-a-new-earthquake-catalogue}}

If you wish to submit your own earthquake catalogue, you can do this on
the ‘\sphinxstyleemphasis{File’} tab of the ‘Earthquakes’ page shown in figure 8.4. The
catalogue must be uploaded in quakeML format. The easiest way to convert
other earthquake catalogue formats to quakeML is using ObsPy. The
earthquakes must have full details of the location and focal mechanism
of all the events you are interested in modelling.

The ObsPy command ‘\sphinxstyleemphasis{readEvents’} can be used to read in events that
are in a range of text based formats (e.g. NDK \& ZMAP). The event data
can then be written to a quakeML file using ‘\sphinxstyleemphasis{writeQuakeML’}. Full
details on how to install ObsPy and access tutorials on ObsPy are given
in appendix 3.

\sphinxincludegraphics{{image54}.png}

\sphinxstylestrong{Figure 8.4:} The input form for bespoke user input earthquake
catalogues.


\section{Submitting a new station catalogue}
\label{\detokenize{Section8:submitting-a-new-station-catalogue}}
New seismic stations and networks can be input in a similar way using
the upload form under the ‘File’ tab in the ‘Stations’ section shown in
figure 8.5.1.

The format of the input station file can be a simple list of station
name, group (i.e. network name), longitude, latitude, depth and meters
buried. Figure 8.5.2 shows an example of the station locations for the
temporary Maule network in Central Chile, which is already loaded into
the platform. Alternatively, the station file can be uploaded as an xml
file by selecting this format in the drop down menu and choosing your
appropriate file.

\sphinxincludegraphics{{image64}.png}

\sphinxstylestrong{Figure 8.5.1:} The input form for new seismic station networks.

\sphinxincludegraphics{{image74}.png}

\sphinxstylestrong{Figure 8.5.2:} The input format for stations and networks to be added
to the VERCE platform.


\chapter{VERCE glossary}
\label{\detokenize{Glossary::doc}}\label{\detokenize{Glossary:verce-glossary}}

\sphinxstylestrong{Workflow} \textendash{} refers to a sequence of jobs that can be submitted. In
the VERCE project we have a number of workflows prepared that can run
you job on a specific high performance computer.

\sphinxstylestrong{HPC} \textendash{} high performance computing. This usually refers to parallel
computers, where a given computational task is spread over many separate
processing cores.

\sphinxstylestrong{iRODS} \textendash{} a suite of data managements software that is embedded within
the VERCE platform, and allows you to easily access your data regardless
of where you submitted your simulation.

\sphinxstylestrong{superMUC} \textendash{} a super computer hosted by the LRZ in Munich, Germany.

\sphinxstylestrong{SCAI} \textendash{} a super computer hosted by CINECA in Italy.

\sphinxstylestrong{DT} \textendash{} the time step of the waveform model.

\sphinxstylestrong{Solver} \textendash{} shorthand for the code that does the forward calculation.
The solver currently hosted in the VERCE platform is SPECFEM.

\sphinxstylestrong{Mesh} \textendash{} the grid over which the wavefield is calculated. The modelled
space is broken up into a grid of points, each with specific seismic
properties (e.g. p-wave velocity, s-wave velocity, seismic attenuation).
The spacing of this grid is able to change especially with depth (due to
the increasing seismic velocity). The structure of these grid points is
referred to as the mesh.

\sphinxstylestrong{Velocity model} \textendash{} this is the seismic velocity model that is input
for an area, and includes the p-wave and s-wave velocities. Most models
in the VERCE portal are 3D, though there are some 2D models (for
subduction zones) and 1D models (global 1D velocity models) available.

\sphinxstylestrong{CPML} \textendash{} ‘convolutional perfectly matched layers’ are a type of
absorbing boundary condition

\sphinxstylestrong{Absorbing Boundary Conditions} \textendash{} are at the edges of the model that
shouldn’t reflect the seismic energy. These boundary conditions are
designed to absorb an incoming wave simulating an infinite medium.

\sphinxstylestrong{CUBIT} \textendash{} an external program used for making meshes for a variety of
scientific and engineering modelling disciplines.

\sphinxstylestrong{TRELIS} \textendash{} the commercial name for the CUBIT package.

\sphinxstylestrong{GeoCubit} \textendash{} a python based program that uses CUBIT command to create
meshes for geographical bodies. Particularly, GeoCubit can produce
meshes that include topography and/or bathymetry.


 \appendix
 \renewcommand{\thechapter}{A\arabic{chapter}}
\chapter{Appendix 1 \textendash{} SPECFEM3D\_Cartesian’s Flags}
\label{\detokenize{Appendix1::doc}}\label{\detokenize{Appendix1:appendix-1-specfem3d-cartesians-flags}}
The input parameters for the code SPECFEM3D\_Cartesian are briefly
described below. Please see the manual of the code for a detailed
description.


\section{Group 0 - Basic}
\label{\detokenize{Appendix1:a1-1-group-0-basic}}
\sphinxincludegraphics[scale=0.8]{{image1}.png}

\sphinxstylestrong{Figure A1.1:} Parameter form for ‘Group 0 - Basic’.

\sphinxstylestrong{NPROC} is the number of processors that the simulation is run on.
This is essentially dependent upon the high performance computer and
workflow you intend to submit your job to.

\sphinxstylestrong{NSTEP} is the number of time steps that you want to run your
simulation for. This should be set so that (NSTEP * DT) is equal to the
time in seconds you want to simulate. So the model setup shown above
will run a simulation of 60 seconds, and provide synthetic seismograms
for 60 seconds after the origin time of the simulated earthquake.

\sphinxstylestrong{DT} is the time step in seconds used in the solver. This must be
small enough to ensure that the simulated waveform is properly sampled
and that the calculations are stable. The equations this is based on are
given in section 8 of this guide. For the meshes and models that are
already available in the portal though, the recommended DT is given in
Figure 4.4 and is the default in the portal.

\sphinxstylestrong{MODEL} allows you to select the velocity model that is used in the
simulation. Leaving this to ‘\sphinxstyleemphasis{default’} will select the 3D velocity
model that is specified in the drop down menu next to ‘\sphinxstyleemphasis{Velocity
Model}’, at the top of the input parameters panel. It is however also
possible to select from a range of 1D models that are pre-loaded into
the solver SPECFEM3D\_Cartesian. (See the code’s manual for all the
available options).

\sphinxstylestrong{GPU\_MODE} allows SPECFEM to be run on high performance computers
that use graphical processing units (GPUs) rather than the more
conventional CPU (central processing unit). All the workflows currently
available on the VERCE platform use CPUs, so you should always leave
this box unchecked.


\section{Group 1 \textendash{} Inverse problem}
\label{\detokenize{Appendix1:a1-2-group-1-inverse-problem}}
In addition to calculating the wavefield from an earthquake source
(referred to as a ‘forward simulation’), SPECFEM can also be used to
calculate the adjoint wavefield, as well as being able to simulate noise
sources for ambient noise tomography applications. These options are
controlled by this group of parameters.

\sphinxincludegraphics{{image2}.png}

\sphinxstylestrong{Figure A1.2:} Parameter form for ‘Group 1 \textendash{} Inverse problem’.

\sphinxstylestrong{SIMULATION\_TYPE} is set to ‘\sphinxstyleemphasis{forward}’ by default to model the
wave-field from an earthquake.

\sphinxstylestrong{NOISE\_TOMOGRAPHY} is set to ‘\sphinxstyleemphasis{earthquake simulation}’ by default
as the noise tomography applications of SPECFEM are not currently
supported within the VERCE platform.

\sphinxstylestrong{SAVE\_FORWARD} is selected if the last step of the wave-field is to
be saved. This enables to back reconstruct the seismic wave-field, but
requires a large amount of storage space and it is not yet supported by
the VERCE platform.


\section{Group 2 \textendash{} UTM projection}
\label{\detokenize{Appendix1:a1-3-group-2-utm-projection}}
As SPECFEM3D\_Cartesian uses, unsurprisingly, Cartesian coordinates, you
must specify the UTM zone that your model falls in. This is described in
more detail in section 8 when we consider uploading new meshes and
models. For the pre-loaded meshes and models though the correct UTM zone
is given by the tables shown in Figures 4.4 and 4.5, and is set
correctly by default when the mesh is selected.

\sphinxincludegraphics{{image3}.png}

\sphinxstylestrong{Figure A1.3:} Parameter form for ‘Group 2 \textendash{} UTM projection’.

\sphinxstylestrong{UTM\_PROJECTION\_ZONE} is where the UTM zone is specified. Only valid
when SUPPRESS\_UTM\_PROJECTION is unchecked (as in our case).

\sphinxstylestrong{SUPPRESS\_UTM\_PROJECTION} is not enabled in the VERCE platform,
meaning the model range must always be specified in geographical
coordinates (not Cartesian coordinates) and the conversion will be done
inside the code.


\section{Group 3 \textendash{} Attenuation}
\label{\detokenize{Appendix1:a1-4-group-3-attenuation}}
In the Earth seismic waves are attenuated by the visco-elastic
deformation as the wave propagates. If we are to gain simulated seismic
waves with a similar amplitude to the recorded waves, we must include
this attenuation in our waveform simulation.

\sphinxincludegraphics[scale=0.8]{{image4}.png}

\sphinxstylestrong{Figure A1.4:} Parameter form for ‘Group 3 \textendash{} Attenuation’.

\sphinxstylestrong{ATTENUATION} controls whether attenuation is incorporated or not.
Turning attenuation on means that extra variables are generated, and
therefore will increase the time taken for the simulation to run and
also the memory requirements.

\sphinxstylestrong{USE\_OLSEN\_ATTENUATION} can be used to define the attenuation model
from the S-wave velocity using the empirical relationship proposed by
Olsen et al. (2003).

\sphinxstylestrong{OLSEN\_ATTENUATION\_RATIO} determines the Olsen’s constant in Olsen’s
empirical relation and should be in the range of 0.02-0.1.

\sphinxstylestrong{MIN\_ATTENUATION\_PERIOD} is the minimum of the attenuation period
range over which we try to mimic a constant Q factor.

\sphinxstylestrong{MAX\_ATTENUATION\_PERIOD} is the maximum of the attenuation period
range over which we try to mimic a constant Q factor.

\sphinxstylestrong{COMPUTE\_FREQ\_BAND\_AUTOMATIC} is used to ignore the above range and
ask the code to compute it automatically based on the estimated
resolution of the mesh.

\sphinxstylestrong{ATTENUATION\_f0\_REFERENCE} is the reference frequency for target
velocity values in the velocity model.


\section{Group 4 \textendash{} Absorbing Boundary Conditions}
\label{\detokenize{Appendix1:a1-5-group-4-absorbing-boundary-conditions}}
Parameters of this group allow to choose between Stacey absorbing
conditions or ‘convolutional perfectly matched layers’ (CPMLs) The last
ones are the most effective and therefore computationally efficient
absorbing boundary conditions and should be considered for all new
meshes that are uploaded. It is especially important that they are used
in models where you are particularly worried about side reflections
(e.g. models where receivers or particularly sources are very close to
the model edge). For a full discussion of the relative merits of the two
methods, please see the SPECFEM3D\_Cartesian manual.

\sphinxincludegraphics[scale=0.8]{{image5}.png}

\sphinxstylestrong{Figure A1.5:} Parameter form for ‘Group 4 \textendash{} Absorbing Boundary
Conditions’.

\sphinxstylestrong{PML\_CONDITIONS} select whether CPMLs are implemented. Please ensure
that Stacey absorbing conditions are unchecked if you do this. If
PML\_CONDITIONS and STACEY\_ABSORBING\_CONDITIONS are both unchecked,
you get a free surface instead.

\sphinxstylestrong{PML\_INSTEAD\_OF\_FREE\_SURFACE} replaces the free surface at the top
of the model with a PML absorbing layer. This can be useful if you are
simulating a deep model, rather than a model that includes the Earth’s
surface.

\sphinxstylestrong{f0\_FOR\_PML} is the dominant frequency of CPML, or the frequency at
which the PML will be the most effective. It should therefore be set to
the dominant frequency of the waveforms being simulated.

\sphinxstylestrong{STACEY\_ABSORBING\_CONDITIONS} is selected to activate
Clayton-Enquist absorbing boundary conditions on the sides and bottom of
the simulated areas. This is designed to prevent artificial reflections
from the model edges from affecting the simulated waveforms.

\sphinxstylestrong{ROTATE\_PML\_ACTIVATE} parameter used to rotate C-PML boundary
conditions by a given angle (not implemented yet)

\sphinxstylestrong{ROTATE\_PML\_ANGLE} parameter used to set the angle by which we want
the C-PML boundary conditions to be rotated (not implemented yet).

\sphinxstylestrong{BOTTOM\_FREE\_SURFACE} is checked to make the bottom surface of the
mesh a free surface instead of absorbing.

\sphinxstylestrong{STACEY\_INSTEAD\_OF\_FREE\_SURFACE} is largely the same as the
‘PML\_INSTEAD\_OF\_FREE\_SURFACE’ option, but the free surface is
replaced with the less effective Clayton-Enquist style absorbing
boundary conditions.


\section{Group 5 \textendash{} Seismograms}
\label{\detokenize{Appendix1:a1-6-group-5-seismograms}}
These parameters control the output of seismograms produced by
SPECFEM3D\_Cartesian.

\sphinxincludegraphics[scale=0.7]{{image6}.png}

\sphinxstylestrong{Figure A1.6:} Parameter form for ‘Group 5 \textendash{} Seismograms’.

\sphinxstylestrong{NTSTEP\_BETWEEN\_OUTPUT\_SEISMOS} controls the frequency (in number
of time steps) that the seismograms are written to disk. Fewer disk
writes will allow the simulation to run quicker, but will also increase
the amount of data that is lost if the code does crash.

\sphinxstylestrong{SAVE\_SEISMOGRAMS\_DISPLACEMENT} is checked if we want to save
displacement in the forward runs (can be checked simultaneously to the
following three flags).

\sphinxstylestrong{SAVE\_SEISMOGRAMS\_VELOCITY} is checked if we want to save velocity
in the forward runs (can be checked simultaneously to the previous and
following two flags).

\sphinxstylestrong{SAVE\_SEISMOGRAMS\_ACCELERATION} is checked if we want to save
acceleration in the forward runs (can be checked simultaneously to the
previous two and following two flags).

\sphinxstylestrong{SAVE\_SEISMOGRAMS\_PRESSURE} is checked if we want to save pressure
in the forward runs (can be checked simultaneously to the previous three
flags). Currently it is implemented in acoustic elements only.

\sphinxstylestrong{USE\_BINARY\_FOR\_SEISMOGRAMS} saves seismograms in binary instead of
ASCII format (binary is smaller but may not be portable between
machines).

\sphinxstylestrong{SU\_FORMAT} outputs seismograms in Seismic Unix format (binary with
240-byte-headers).

\sphinxstylestrong{WRITE\_SEISMOGRAMS\_BY\_MASTER} decides if master process writes all
the seismograms or if all processes do it in parallel.

\sphinxstylestrong{SAVE\_ALL\_SEISMOS\_IN\_ONE\_FILE} saves all seismograms in one large
combined file instead of one file per seismogram to avoid overloading
shared non-local file systems such as LUSTRE or GPFS for instance.

\sphinxstylestrong{USE\_TRICK\_FOR\_BETTER\_PRESSURE} allows to use a trick to increase
accuracy of pressure seismograms in fluid (acoustic) elements (see
SPECFEM manual for details).


\section{Group 6 \textendash{} Sources}
\label{\detokenize{Appendix1:a1-7-group-6-sources}}
The VERCE platform is very much configured to simulate earthquake
sources. However there are other types of seismic sources such as active
sources, explosions or impacts that you may want to simulate. This can
be done using the options described below.

\sphinxincludegraphics[scale=0.8]{{image7}.png}

\sphinxstylestrong{Figure A1.7:} Parameter form for ‘Group 6 \textendash{} Sources’.

\sphinxstylestrong{USE\_FORCE\_POINT\_SOURCE} simulates a force point source (e.g.
impact source) rather than an earthquake source. If you are using this
option the source must be defined in a FORCESOLUTUION file, rather than
in the CMT solution convention used for earthquake sources. See the
SPECFEM manual for full details. This option is not yet implemented in
the VERCE portal.

\sphinxstylestrong{USER\_RICKER\_TIME\_FUNCTION} this inputs the source as a Ricker
wavelet, rather than the default delta/gaussian function that is
designed to represent the slip on a fault during an earthquake. Again
this option is useful for simulating non-earthquake seismic sources.

\sphinxstylestrong{USE\_EXTERNAL\_SOURCE\_FILE} is checked to use an external source
time function defined by an input file. This option is not yet
implemented in the VERCE portal.

\sphinxstylestrong{USE\_SOURCE\_ENCODING} determines source encoding factor +/-1
depending on the sign of moment tensor (for acoustic simulations only).

\sphinxstylestrong{PRINT\_SOURCE\_TIME\_FUNCTION} outputs the source time function input
to the simulation as a text file.


\section{Group 7 \textendash{} Visualisation}
\label{\detokenize{Appendix1:a1-8-group-7-visualisation}}
One of the outputs of SPECFEM which can be requested through the VERCE
platform is a movie of the waveform simulation. This is usually output
on the surface topography of the model, but can be built for the whole
3D volume. This last option is extremely demanding on memory though, and
not recommended for normal simulations.

\sphinxincludegraphics[scale=0.8]{{image8}.png}

\sphinxstylestrong{Figure A1.8:} Parameter form for ‘Group 7 \textendash{} Visualisation’.

\sphinxstylestrong{CREATE\_SHAKEMAP} creates a map of peak ground velocity for the area
modelled.

\sphinxstylestrong{MOVIE\_SURFACE} sets the output movie for just the surfaces you
define in MOVIE\_TYPE.

\sphinxstylestrong{MOVIE\_TYPE} selects whether the surface movies and shake-maps are
generated for the top surface of the model (topography + oceans) only,
or for all external faces of the mesh (i.e. topography + vertical edges
+ bottom).

\sphinxstylestrong{MOVIE\_VOLUME} allows 3D snapshots of the entire model volume to be
output. This would allow the entire wave-field to be imaged, which could
be useful. But this would also be hugely demanding on memory and so
should be left unchecked by default.

\sphinxstylestrong{SAVE\_DISPLACMENT} saves displacement in the movie snapshots, rather
than the default, which is to save velocity for the movie.

\sphinxstylestrong{USE\_HIGHRES\_FOR\_MOVIES} saves the wave-field values for movies at
all the grid points so that the resolution of the movie is the same as
the resolution of the model. Selecting this option requires a large
amount of memory, so should not be selected by default.

\sphinxstylestrong{NTSTEPS\_BETWEEN\_FRAMES} sets the number of time steps between
snapshots of the wave-field. The spacing of the frames in seconds is
given by (NTSTEPS\_BETWEEN\_FRAMES*DT).

\sphinxstylestrong{HDUR\_MOVIE} is the half duration of the source time function for the
movie simulation.


\section{Group 8 \textendash{} Adjoint Kernel Options}
\label{\detokenize{Appendix1:a1-9-group-8-adjoint-kernel-options}}
Beyond forward simulations, SPECFEM3D\_Cartesian allows for the
simulation of adjoint wave-fields useful for adjoint travel time
tomography procedures (Tromp et al., 2005). These simulations are
controlled by the options described below.

\sphinxincludegraphics[scale=0.8]{{image9}.png}

\sphinxstylestrong{Figure A1.9:} Parameter form for ‘Group 8 \textendash{} Adjoint Kernel
Options’.

\sphinxstylestrong{NTSTEP\_BETWEEN\_READ\_ADJSRC} interval in time steps for reading
adjoint traces.

\sphinxstylestrong{ANISOTROPIC\_KL} allows to compute anisotropic kernels in crust and
mantle instead of the default, which is to compute isotropic kernels.

\sphinxstylestrong{SAVE\_TRANSVERSE\_KL} allows to compute transverse isotropic kernels
rather than fully anisotropic kernels.

\sphinxstylestrong{APPROXIMATE\_HESS\_KL} outputs approximate Hessian for
preconditioning.

\sphinxstylestrong{SAVE\_MOHO\_MESH} saves Moho mesh and computes Moho boundary kernels.


\section{Group 9 \textendash{} Advanced}
\label{\detokenize{Appendix1:a1-10-group-9-advanced}}
There are a large amount of other functions within SPECFEM3D\_Cartesian
that can be altered using the VERCE platform. A brief description of
these functions is given below, but in most cases if you intend to use
these advanced options you should also refer to the SPECFEM m


\begin{savenotes}\sphinxattablestart
\centering
\begin{tabulary}{\linewidth}[t]{T T}

\sphinxincludegraphics{{image10}.png}
&
\sphinxincludegraphics{{image11}.png}
\\

\end{tabulary}
\par
\sphinxattableend\end{savenotes}

\begin{center}\sphinxstylestrong{Figure A1.10:} Parameter form for ‘Group 9 \textendash{} Advanced’.
\end{center}
\sphinxstylestrong{NSTEPS\_BETWEEN\_OUTPUT\_INFO} controls the frequency that
information about a running simulation is output to a log file.

\sphinxstylestrong{NGNOD} controls the number of nodes for each element of the
hexahedral mesh. For all meshes loaded into the VERCE platform and all
meshes created using CUBIT this should be left at the default value of
8.

\sphinxstylestrong{APROXIMATE\_OCEAN\_LOAD} is a relatively computationally cheap method
of modelling the effect of oceans on the wave-field. It is however only
effective at relatively low frequencies (periods of 20 seconds and
longer). For higher frequencies if the effects of the water column are
to be modelled, the ocean must be included in the mesh itself.

\sphinxstylestrong{TOPOGRAPHY} is only needed if the ‘APROXIMATE\_OCEAN\_LOAD’ option
above is selected, and reads in the topography/bathymetry files needed
to define that surface.

\sphinxstylestrong{ANISOTROPY} is selected if you want to include seismic anisotropy.
You will also need to provide an anisotropy model to include this, and
this has not been done for any of the pre-loaded meshes and models, and
so cannot be selected for these cases.

\sphinxstylestrong{GRAVITY} is selected if you want to include gravity in your
simulation. It is effective only at very long periods.

\sphinxstylestrong{TOMOGRAPHY\_PATH} is the directory in which the tomography files are
stored for using external tomographic Earth models. For simulations with
the VERCE portal it is not required to set this parameter.

\sphinxstylestrong{SAVE\_MESH\_FILES} saves mesh files in a ‘\sphinxstyleemphasis{Paraview’} format for
later use.

\sphinxstylestrong{LOCAL\_PATH} is the directory in which the files for the partitioned
mesh will be written. For simulations with the VERCE portal it is not
required to set this parameter.

\sphinxstylestrong{SEP\_MODEL\_DIRECTORY} should be set if you are using a SEP model
(oil-industry format). This option is not yet implemented in the VERCE
portal.

\sphinxstylestrong{ADIOS\_ENABLED} is checked to enable ADIOS. If it is not checked,
subsequent ADIOS parameters will not be considered. This option is not
yet implemented in the VERCE portal.

\sphinxstylestrong{ADIOS\_FOR\_DATABASES} is checked to use ADIOS for xmeshfem3D output
and xgenerate\_database input.

\sphinxstylestrong{ADIOS\_FOR\_MESH} is checked to use ADIOS for generated databases.

\sphinxstylestrong{ADIOS\_FOR\_FORWARD\_ARRAYS} is checked to read and write forward
arrays using ADIOS.

\sphinxstylestrong{ADIOS\_FOR\_KERNELS} is checked to produce ADIOS kernels that can
later be visualized with the ADIOS version of combine\_vol\_data.

\sphinxstylestrong{USE\_LDDRK, INCREASE\_CFL\_FOR\_LDDRK,
RATIO\_BY\_WHICH\_TO\_INCREASE\_IT} are the parameters to set up the
LDDRK time scheme. This option is not yet implemented into the VERCE
portal. See the manual of SPECFEM for details.

\sphinxstylestrong{OUTPUT\_ENERGY} allows to plot energy curves, for instance to monitor
how CPML absorbing layers behave. This option is turned off by default
since it is a bit expensive.

\sphinxstylestrong{NTSTEP\_BETWEEN\_OUTPUT\_ENERGY} controls the interval of time steps
between the energy computation.

\sphinxstylestrong{NUMBER\_OF\_SIMULTANEOUS\_RUNS} allows to simultaneously run (in an
embarrassingly-parallel fashion) multiple earthquake simulations each
with the same number of cores. This option is not yet implemented in the
VERCE portal.

\sphinxstylestrong{BROADCAST\_SAME\_MESH\_AND\_MODEL} allows to broadcast the same mesh
and velocity model to multiple events in case of
NUMBER\_OF\_SIMULTANEOUS\_RUNS\textgreater{}1. This option is not yet implemented in
the VERCE portal.


\chapter{Appendix 2 \textendash{} SPECFEM3D\_GLOBE’s Flags}
\label{\detokenize{Appendix2:appendix-2-specfem3d-globes-flags}}\label{\detokenize{Appendix2::doc}}
The input parameters for the code of SPECFEM3D\_GLOBE are briefly
described below. For a detailed description please consult the
SPECFEM3D\_GLOBE manual.


\section{Group 0 - Basic}
\label{\detokenize{Appendix2:a2-1-group-0-basic}}
\sphinxincludegraphics{{image12}.png}

\sphinxstylestrong{Figure A2.1:} Parameter form for ‘Group 0 - Basic’.

\sphinxstylestrong{NPROC} is the number of processors that the simulation will run on.
This is essentially dependent upon the high-performance computer and
workflow you intend to submit your job to.

\sphinxstylestrong{RECORD\_LENGTH\_IN\_MINUTES} is the time in minutes you want to run
the simulation for.

\sphinxstylestrong{MODEL} is the velocity model to be used in the simulation. There is a
range of models pre-loaded into the solver SPECFEM3D\_GLOBE. (See the
code’s manual for all the available options).

\sphinxstylestrong{GPU\_MODE} allows SPECFEM to be run on high performance computers
that use graphical processing units (GPUs) rather than the more
conventional CPU (central processing unit). All the workflows currently
available on the VERCE platform use CPUs, so you should always leave
this box unchecked.


\section{Group 1 \textendash{} Inverse Problem}
\label{\detokenize{Appendix2:a2-2-group-1-inverse-problem}}
\sphinxincludegraphics{{image21}.png}

\sphinxstylestrong{Figure A2.2:} Parameter form for ‘Group 1 \textendash{} Inverse Problem’.

\sphinxstylestrong{SIMULATION\_TYPE} is set to ‘\sphinxstyleemphasis{forward}’ by default to model the
wave-field from an earthquake.

\sphinxstylestrong{NOISE\_TOMOGRAPHY} is set to ‘\sphinxstyleemphasis{earthquake simulation}’ by default
as the noise tomography applications of SPECFEM are not currently
supported within the VERCE platform.

\sphinxstylestrong{SAVE\_FORWARD} is selected if the last step of the wave-field is to
be saved. This enables to back reconstruct the seismic wave-field, but
requires a large amount of storage space and it is not yet supported by
the VERCE platform.


\section{Group 2 \textendash{} Simulation Area}
\label{\detokenize{Appendix2:a2-3-group-2-simulation-area}}
\sphinxincludegraphics{{image31}.png}

\sphinxstylestrong{Figure A2.3:} Parameter form for ‘Group 2 \textendash{} Simulation Area’.

\sphinxstylestrong{ANGULAR\_WIDTH\_XI\_IN\_DEGREES} is the width of one side of the
chunk in degrees.

\sphinxstylestrong{ANGULAR\_WIDTH\_ETA\_IN\_DEGREES} is the width of the second side of
the chunk in degrees.

\sphinxstylestrong{CENTER\_LATITUDE\_IN\_DEGREES} is the latitude of centre of the chunk
in degrees.

\sphinxstylestrong{CENTER\_LONGITUDE\_IN\_DEGREES} is the longitude of centre of the
chunk in degrees.

\sphinxstylestrong{GAMMA\_ROTATION\_AZIMUTH} defines the rotation angle of the chunk
about its centre measured counter clockwise from due North in degrees.

\sphinxstylestrong{OCEANS} can be selected if the effect of the oceans on seismic wave
propagation should be incorporated based upon the approximate treatment
discussed in Komatitsch and Tromp (2002).

\sphinxstylestrong{ELLIPTICITY} can be selected if the mesh should make the Earth model
elliptical in shape according to Clairaut’s equation.

\sphinxstylestrong{TOPOGRAPHY} can be selected if topography and bathymetry should be
incorporated based upon model ETOPO4.

\sphinxstylestrong{GRAVITY} can be selected if self-gravitation should be incorporated
in the Cowling approximation.

\sphinxstylestrong{ROTATION} can be selected if the Coriolis effect should be
incorporated. Turning this feature on is relatively cheap numerically.

\sphinxstylestrong{ATTENUATION} can be selected if attenuation should be incorporated.

\sphinxstylestrong{ABSORBING\_CONDITIONS} is selected only for regional simulations.


\section{Group 3 \textendash{} Mesh Parameters}
\label{\detokenize{Appendix2:a2-4-group-3-mesh-parameters}}
\sphinxincludegraphics{{image41}.png}

\sphinxstylestrong{Figure A2.4:} Parameter form for ‘Group 3 \textendash{} Mesh Parameters’.

\sphinxstylestrong{NCHUNKS} is the number of chunks.

\sphinxstylestrong{NEX\_XI} is the number of elements at the surface along the xi side
of a chunk.

\sphinxstylestrong{NEX\_ETA} is the number of elements at the surface along the eta side
of a chunk.

\sphinxstylestrong{NPROC\_XI} is the number of MPI processors along the xi side of a
chunk.

\sphinxstylestrong{NPROC\_ETA} is the number of MPI processors along the eta side of a
chunk.


\section{Group 4 \textendash{} Adjoint Kernel Options}
\label{\detokenize{Appendix2:a2-5-group-4-adjoint-kernel-options}}
\sphinxincludegraphics{{image51}.png}

\sphinxstylestrong{Figure A2.5:} Parameter form for ‘Group 4 \textendash{} Adjoint Kernel
Options’.

\sphinxstylestrong{READ\_ADJSRC\_ASDF} can be selected to use ASDF format for reading
the adjoint sources.

\sphinxstylestrong{ANISOTROPIC\_KL} can be used to compute anisotropic kernels in crust
and mantle.

\sphinxstylestrong{SAVE\_TRANSVERSE\_KL\_ONLY} can be used to output only transverse
isotropic kernels rather than fully anisotropic kernels when
\sphinxstylestrong{ANISOTROPIC\_KL} above is selected.

\sphinxstylestrong{APPROXIMATE\_HESS\_KL} can be used to output approximate Hessian in
crust mantle region.

\sphinxstylestrong{USE\_FULL\_TISO\_MANTLE} can be used to force transverse isotropy for
all mantle elements.

\sphinxstylestrong{SAVE\_SOURCE\_MASK} can be used to output kernel mask to zero out
source region to remove large values near the sources in the sensitivity
kernels.

\sphinxstylestrong{SAVE\_REGULAR\_KL} can be used to output kernels on a regular grid
instead of on the GLL mesh points.


\section{Group 5 - Movie}
\label{\detokenize{Appendix2:a2-6-group-5-movie}}

\begin{savenotes}\sphinxattablestart
\centering
\begin{tabulary}{\linewidth}[t]{T T}


\sphinxincludegraphics{{image61}.png}
&
\sphinxincludegraphics{{image71}.png}
\\
\end{tabulary}
\par
\sphinxattableend\end{savenotes}

\begin{center}\sphinxstylestrong{Figure A2.6:} Parameter form for ‘Group 5 - Movie’.
\end{center}
\sphinxstylestrong{MOVIE\_SURFACE} creates a movie of seismic wave propagation on the
Earth’s surface.

\sphinxstylestrong{MOVIE\_VOLUME} creates a movie of seismic wave propagation in the
Earth’s interior.

\sphinxstylestrong{MOVIE\_COARSE} saves movie only at corners of elements.

\sphinxstylestrong{NTSTEP\_BETWEEN\_FRAMES} determines the number of timesteps between
two movie frames.

\sphinxstylestrong{HDUR\_MOVIE} determines the half duration of the source time function
for the movie simulations.

\sphinxstylestrong{MOVIE\_VOLUME\_TYPE} allows you to select movie volume type option
where 1=strain, 2=time integral of strain, 3=µ*time integral of strain,
4= saves the trace and deviatoric stress in the whole volume,
5=displacement, 6=velocity.

\sphinxstylestrong{MOVIE\_TOP\_KM/MOVIE\_BOTTOM\_KM} defines depth below the surface in
kilometres.

\sphinxstylestrong{MOVIE\_WEST\_DEG} refers to longitude, degrees West.

\sphinxstylestrong{MOVIE\_EAST\_DEG} refers to longitude, degrees East.

\sphinxstylestrong{MOVIE\_NORTH\_DEG} refers to latitude, degrees North.

\sphinxstylestrong{MOVIE\_SOUTH\_DEG} refers to latitude, degrees South.

\sphinxstylestrong{MOVIE\_START} denotes movie start time.

\sphinxstylestrong{MOVIE\_STOP} denotes movie end time.


\section{Group 6 - Sources}
\label{\detokenize{Appendix2:a2-7-group-6-sources}}
\sphinxincludegraphics{{image81}.png}

\sphinxstylestrong{Figure A2.7:} Parameter form for ‘Group 6 - Sources’.

\sphinxstylestrong{NTSTEP\_BETWEEN\_READ\_ADJSRC} refers to the number of adjoint
sources read in each time for an adjoint simulation.

\sphinxstylestrong{PRINT\_SOURCE\_TIME\_FUNCTION} prints information about the source
time function in the file
OUTPUT\_FILES/plot\_source\_time\_function.txt.


\section{Group 7 - Seismograms}
\label{\detokenize{Appendix2:a2-8-group-7-seismograms}}
\sphinxincludegraphics{{image91}.png}

\sphinxstylestrong{Figure A2.8:} Parameter form for ‘Group 7 - Seismograms’.

\sphinxstylestrong{NTSTEP\_BETWEEN\_OUTPUT\_SEISMOS} specifies the interval at which
synthetic seismograms are written in the LOCAL\_PATH directory.

\sphinxstylestrong{OUTPUT\_SEISMOS\_FORMAT} allows you to select the output format for
the seismograms such as ASCII, SAC\_ALPHANUM, SAC\_BINARY and ASDF.

\sphinxstylestrong{ROTATE\_SEISMOGRAMS\_RT} can be selected to have radial (R) and
transverse (T) horizontal components of the synthetic seismograms.

\sphinxstylestrong{WRITE\_SEISMOGRAMS\_BY\_MASTER} can be selected to have all the
seismograms written by the master.

\sphinxstylestrong{SAVE\_ALL\_SEISMOS\_IN\_ONE\_FILE} saves all seismograms in one large
combined file instead of one file per seismogram.

\sphinxstylestrong{USE\_BINARY\_FOR\_LARGE\_FILE} can be selected to use binary instead
of ASCII for that large file.

\sphinxstylestrong{RECEIVERS\_CAN\_BE\_BURIED} can be used to accommodate stations with
instruments that are buried, i.e., the solver will calculate seismograms
at the burial depth specified in the STATIONS file.


\section{Group 8 - Advanced}
\label{\detokenize{Appendix2:a2-9-group-8-advanced}}

\begin{savenotes}\sphinxattablestart
\centering
\begin{tabulary}{\linewidth}[t]{T T}


\sphinxincludegraphics{{image101}.png}
&
\sphinxincludegraphics{{image111}.png}
\\
\end{tabulary}
\par
\sphinxattableend\end{savenotes}

\begin{center}\sphinxstylestrong{Figure A2.8:} Parameter form for ‘Group 8 - Advanced’.
\end{center}
\sphinxstylestrong{PARTIAL\_PHYS\_DISPERSION\_ONLY} or \sphinxstylestrong{UNDO\_ATTENUATION} can be used
to undo attenuation for sensitivity kernel calculations or forward runs
with SAVE\_FORWARD

\sphinxstylestrong{MEMORY\_INSTALLED\_PER\_CORE\_IN\_GB} is used to set the amount of
memory installed per core in Gigabyte.

\sphinxstylestrong{PERCENT\_OF\_MEM\_TO\_USE\_PER\_CORE} can be used to set percentage
of memory to use per core for arrays to undo attenuation, keeping in
mind that you need to leave some memory available for the GNU/Linux
system to run.

\sphinxstylestrong{EXACT\_MASS\_MATRIX\_FOR\_ROTATION} can be selected if you are
interested in precise effects related to rotation.

\sphinxstylestrong{USE\_LDDRK} can be used for LDDRK high-order time scheme instead of
Newmark.

\sphinxstylestrong{INCREASE\_CFL\_FOR\_LDDRK} can be used to increase CFL for LDDRK.

\sphinxstylestrong{RATIO\_BY\_WHICH\_TO\_INCREASE\_IT} determines the ratio by which to
increase CFL.

\sphinxstylestrong{SAVE\_MESH\_FILES} can be used to save AVS, OpenDX, or ParaView mesh
files for subsequent viewing.

\sphinxstylestrong{NUMBER\_OF\_RUNS} refers to the number of stages in which the
simulation will be completed, e.g. 1 corresponds to a run without
restart files.

\sphinxstylestrong{NUMBER\_OF\_THIS\_RUN} can be used if you choose to perform the run
in stages in which you need to tell the solver what stage run to
perform.

\sphinxstylestrong{NUMBER\_OF\_SIMULTANEOUS\_RUNS} adds the ability to run several
calculations (several earthquakes) in an embarrassingly-parallel fashion
from within the same run.

\sphinxstylestrong{BROADCAST\_SAME\_MESH\_AND\_MODEL} allows to read the mesh and model
files from a single run in the beginning and broadcast them to all the
others (if the mesh and the model are the same for all simultaneous
runs).

\sphinxstylestrong{USE\_FAILSAFE\_MECHANISM} can be used to terminate all the runs or
let the others finish using a fail-safe mechanism if one or a few of
simultaneous runs fail.

\sphinxstylestrong{GPU\_RUNTIME} can only be used if GPU\_MODE is selected.

\sphinxstylestrong{GPU\_PLATFORM} filters on the platform in OpenCL.

\sphinxstylestrong{GPU\_DEVICE} filters on the device name in OpenCL.

\sphinxstylestrong{ADIOS\_ENABLED} and all the other ADIOS flags enable the use of ADIOS
library for I/Os.


\chapter{Appendix 3  \textendash{} Using ObsPy}
\label{\detokenize{Appendix3::doc}}\label{\detokenize{Appendix3:appendix-3-using-obspy}}
ObsPy is a python based seismology toolbox, which can be used to deal
with seismic waveform data and earthquake catalogue information. The
toolbox can read a large range of seismic data formats and perform a
large range of processing applications such as filtering the data, and
misfit calculation.

Many of the pre- and post-processing applications within the portal use
python and ObsPy. The toolbox can also be used to format input data, for
instance producing an earthquake catalogue in quakeML format.

\sphinxincludegraphics{{image13}.png}

\sphinxstylestrong{Figure A3.1:} The ObsPy logo

\sphinxstylestrong{A3.1 Installing Python and ObsPy}
\begin{enumerate}
\item {} 
\sphinxstylestrong{Installing Python:} Depending on your operating system this can be
installed through the software manager. Alternatively the latest
version of python can be installed here,
\begin{quote}

\sphinxurl{https://www.python.org/downloads}

You will also require to specific python libraries that can be
installed from GitHub as described below.
\end{quote}

\end{enumerate}
\begin{enumerate}
\item {} 
\sphinxstylestrong{Installing anaconda:} if you do not have super user privalages on
your machine then both ObsPy and dispel4py (see appendix two) can be
installed using Anaconda, a package to manage and deploy Python
packages. Anaconda can be installed on mac and Linux operating
systems as described here,
\begin{quote}

\sphinxurl{http://docs.continuum.io/anaconda/install.html}

\sphinxstylestrong{Installing ObsPy}: Make sure you have a C and fortran compile
installer. You can then install ObsPy with the command,

conda install -c obspy obspy

or, if you have not installed anaconda,

sudo pip install obspy

Full instructions can be found at

\sphinxurl{https://github.com/obspy/obspy/wiki\#installation}
\end{quote}

\end{enumerate}

\sphinxstylestrong{A3.2 Other dependencies you need to run ObsPy are listed below;}

\sphinxstylestrong{numPy} \textendash{} a toolbox for doing numerical applications in python.

sudo apt-get install python-numpy

\sphinxstylestrong{SciPy} \textendash{} a generic scientific programming toolbox for applications in
python.

sudo apt-get install python-scipy

Additionally, in order to do the plotting parts of the tutorial below
you will need to install the following;

\sphinxstylestrong{Matplotlib} \textendash{} a toolbox for creating plots and figures for python
applications. Figures are customised very easily and intuitively, and
can be exported in a number of different. Files can also be exported
easily to a .mat file for Matlab users.

sudo apt-get install python-matplotlib

\sphinxstylestrong{A3.3 Using ObsPy}

A full online python tutorial, that covers everything from a basic
introduction to ObsPy, up to more advanced applications such as
developing an automated processing workflow, can be found at the link
below;

\sphinxurl{http://docs.obspy.org/tutorial/}

\sphinxincludegraphics{{image2}.jpg}

\sphinxstylestrong{Figure A3.2:} Z-component data plotted using ObsPy. Image
re-produced from

\sphinxurl{http://docs.obspy.org/tutorial/code\_snippets/reading\_seismograms.html}


\chapter{Appendix 4 \textendash{} using dispel4py}
\label{\detokenize{Appendix4::doc}}\label{\detokenize{Appendix4:appendix-4-using-dispel4py}}
Dispel4Py is a python library that allows workflows to be written that
can easily scale to different sizes of computational resource, ranging
from your own laptop, to a large parallel supercomputer. This means that
you can devolve and test your workflow locally on your own machine
before transferring the workflow to a much larger computer to process a
large amount of data.

In seismology this could be particularly useful for performance
calculations with very large data sets, such as noise correlation.
Dispel4Py is used widely in the VERCE portal in the pre- and
post-processing workflows that are implemented there. This toolbox is
especially useful for any seismologist who is looking to speed up a data
application.

\sphinxincludegraphics{{image14}.png}

\sphinxstylestrong{Figure A4.1:} The Dispel4Py logo.


\section{Installing Dispel4Py}
\label{\detokenize{Appendix4:a4-1-installing-dispel4py}}
Firstly please ensure that you have an up to data version of Python
installed as described in appendix one. You can then install Dispel4Py
and its dependencies (such as MPI) as described below.
\begin{enumerate}
\item {} 
\sphinxstylestrong{Installing dispel4py:} Dispel4py can be installed using the
command,
\begin{quote}

pip install dispel4py

Full instructions for installing dispel4py can be found at,
\begin{quote}

\sphinxurl{https://github.com/dispel4py/dispel4py}
\end{quote}
\end{quote}

\end{enumerate}
\begin{enumerate}
\item {} 
\sphinxstylestrong{Installing MPI and mpi4py (optional):} If you wish to explore the
parallel mapping of dispel4py to MPI you may want to install these on
your machine. Different implementations of MPI are available, for
example Open MPI or MPICH2. Depending on your operating system MPI
can be installed through the software manager. You can then install
mpi4py with the command:
\begin{quote}

pip install mpi4py
\end{quote}

\end{enumerate}

Full instructions can be found at:

\sphinxurl{http://mpi4py.scipy.org/docs/usrman/install.html}


\section{Using Disple4Py}
\label{\detokenize{Appendix4:a4-2-using-disple4py}}
A full tutorial describing how to create workflows and perform
sequential and parallel processing using Dispel4Py can be found at the
link below.

\sphinxurl{http://www2.epcc.ed.ac.uk/~amrey/VERCE/Dispel4Py/}

For more information on the basic principles of Dispel4Py please see the
following presentation.

\sphinxurl{http://www.verce.eu/Training/UseVERCE/2015-7-VERCE-dispel4py-basic.pdf}

An advanced introduction for those who wish use Dispel4Py to create
workflows for their own applications can be found at the link below.

\sphinxurl{http://www.verce.eu/Training/UseVERCE/2015-7-VERCE-dispel4py-advanced.pdf}

\sphinxincludegraphics{{image21}.jpg}

\sphinxstylestrong{Figure A4.2:} An example of a workflow to perform a cross
correlation of seismic noise data


\renewcommand{\indexname}{Index}
\printindex
\end{document}



\renewcommand{\indexname}{Index}
\printindex
\end{document}